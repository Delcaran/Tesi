\chapter{Argomenti correlati}

\section{Deflazione}\label{deflazione}

Come si è detto, Bitcoin è un sistema monetario privo di inflazione, in quanto nessuna entità centrale può emettere moneta.
Esiste però una elevatissima propensione alla deflazione\footnote{Diminuzione generale del livello dei prezzi.} la quale potrebbe avere serie conseguenze per la sicurezza dell'intera rete.
Dato il loro estremo valore in rapporto alle monete circolanti, le BTC tendono ad essere accumulate invece che spese sparendo quindi dal sistema dal punto di vista economico. Stessa fine fanno le monete per le quali è stata persa la chiave privata che permetteva loro di essere spese.
Mano a mano che le BTC spariscono dalla circolazione (per volontà o incidente), diminuisce il volume delle transazioni, quindi anche il numero di blocchi generati nell'unità di tempo e di conseguenza il numero di nuove monete introdotte in circolazione, il quale viene già dimezzato ad intervalli programmati in modo da generare in assoluto al massimo 21 milioni di BTC.
L'attrattiva della creazione di nuovi blocchi diventa quindi via via sempre più ridotta, in quanto non solo ci saranno meno BTC generate, ma anche meno tasse da parte degli utenti a causa delle minori transazioni.
Se la circolazione di BTC si riduce troppo, potrebbe verificarsi una perdita di interesse nel sistema, il che si traduce in un numero sempre minore di miners e quindi di ``verificatori'' per le transazioni, fino al punto in cui il sistema è diventato troppo debole per potersi difendere da fork nella blockchain o altri attacchi.\\\\
Purtroppo non esistono contromisure pratiche per il rischio di deflazione, e l'unico modo per rendere Bitcoin resistente a tale pericolo consiste nel rendere consapevoli gli utenti ed in particolare i minatori di quanto importante sia il loro contributo per la salvaguardia del loro stesso denaro.\\
Si può però imparare da Bitcoin per una futura moneta elettronica che sia resistente alla deflazione integrando un sistema di feedback per controllare il tasso di mining globale, in modo che rifornire il sistema nel caso la circolazione diminuisca.


\chapter{Argomenti correlati}

\section{Scripting}\label{transaction-scripting}

Le transazioni Bitcoin contengono un potenziale nascosto che non viene ancora sfruttato dalle attuali implementazioni del client.
Alcuni campi delle transazioni possono infatti contenere piccoli script scritti in un linguaggio di programmazione Forth-like basato su stack valutato da sinistra verso destra, e di fatto nelle implementazioni attuali contengono script per la verifica della correttezza delle transazioni. Il linguaggio di scripting è descritto in \cite{bitcoin-scripting-api} ed offre funzioni crittografiche quali \verb|SHA1|, che rimpiazza l'elemento in cima allo stack con il suo hash, e \verb|CHECKSIG| che estrae una chiave pubblica ECDSA e la relativa firma dallo stack, verifica la firma per un messaggio implicitamente definito dai dati della transazione e carica il risultato (vero o falso) nello stack.\\\\
Per capire meglio, il seguente listato mostra il contenuto una transazione standard (transazione \verb|pay-to-pubkey-hash|) decodificato dall'esadecimale\footnote{Un esempio di messaggio di transazione non decodificato (così come una chiara rappresentazione di tutte le strutture dati usate dal protocollo) è disponibile in \url{https://en.bitcoin.it/wiki/Protocol_specification\#Transaction_Verification} } ed espresso in formato JSON\footnote{I dati qui sono stati troncati per brevità, ma la transazione è realmente esistente ed è visibile all'indirizzo \url{http://blockexplorer.com/rawtx/99383066a5140b35b93e8f84ef1d40fd720cc201d2aa51915b6c33616587b94f}}:

\lstinputlisting[tabsize=2, numbers=left]{src/transaction.json}

I campi della transazione hanno il seguente contenuto:

\begin{description}
    \item[\PVerb{hash}] è l'hash della transazione calcolato secondo appositi flag (vedere più avanti \ref{contratti}) e utilizzato per identificarla.
    \item[\PVerb{ver}] indica la versione del protocollo Bitcoin a cui questa transazione si riferisce.
    \item[\PVerb{vin_sz}] è il numero di input presenti.
    \item[\PVerb{vout_sz}] è il numero di output.
    \item[\PVerb{lock_time}] permette di rendere non pagabile la transazione (e quindi di modificarla) fino ad un certo istante nel futuro. Maggiori dettagli nella sezione \ref{contratti}.
    \item[\PVerb{size}] la dimensione in byte della transazione.
    \item[\PVerb{in}] array degli input.\begin{description}
      \item[\PVerb{prev_out}] contiene i riferimenti da cui prelevare le BTC:\begin{description}
            \item[\PVerb{hash}] è l'identificativo della transazione a cui fare riferimento.
            \item[\PVerb{n}] è l'indice dell'output della transazione di riferimento da cui prelevare le BTC.
        \end{description}
      \item[\PVerb{scriptSig}] contiene una porzione dello script usato per verificare la transazione. In questo caso è nella forma \verb|<sig> <pubKey>| dove \verb|sig| è firma ECDSA dell'autore della transazione mentre \verb|pubKey| è la corrispondente chiave pubblica.
      \end{description}
    \item[\PVerb{out}] array degli output.\begin{description}
      \item[\PVerb{value}] ammontare di questo output.
      \item[\PVerb{scriptPubKey}] seconda porzione dello script. Si trova nella forma \verb|OP_DUP| \verb|OP_HASH160| \verb|<pubKeyHash>| \verb|OP_EQUALVERIFY| \verb|OP_CHECKSIG| dove \verb|pubKeyHash| è l'indirizzo del destinatario della transazione mentre le altre voci sono istruzioni del linguaggio di scripting.
    \end{description}
\end{description}

La verifica di un input di tale transazione avviene secondo la procedura illustrata nella tabella \ref{table:pay-to-pubkey-hash}, in cui la transazione da verificare verrà identificata come $\tau_1$, mentre la transazione precedente a cui si fa riferimento sarà $\tau_0$:

\begin{table}
  \centering
  \label{table:pay-to-pubkey-hash}
  \begin{tabular}{m{0.3\linewidth-2\tabcolsep} | m{\dimexpr 0.3\linewidth-2\tabcolsep} | m{\dimexpr 0.3\linewidth-2\tabcolsep}}
    \hline
    Stack&Script&Descrizione \\
    \hline
    \verb||&\verb|<sig>| \verb|<pubKey>| \verb|OP_DUP| \verb|OP_HASH160| \verb|<pubKeyHash>| \verb|OP_EQUALVERIFY| \verb|OP_CHECKSIG|&Unisco il campo \verb|scriptSig| di $\tau_1$ con il campo \verb|scriptPubKey| del relativo output di $tau_0$ creando lo script da eseguire  \\ \hline
    \verb|<sig>| \verb|<pubKey>|&\verb|OP_DUP| \verb|OP_HASH160| \verb|<pubKeyHash>| \verb|OP_EQUALVERIFY| \verb|OP_CHECKSIG|&Carico nello stack la firma dell'autore di $\tau_1$ e la relativa chiave pubblica \\ \hline
    \verb|<sig>| \verb|<pubKey>| \verb|<pubKey>|&\verb|OP_HASH160| \verb|<pubKeyHash>| \verb|OP_EQUALVERIFY| \verb|OP_CHECKSIG|&Duplico la chiave pubblica in cima allo stack. Si dovrà utilizzare due volte: prima per verificare che l'autore di $\tau_1$ sia il destinatario di $\tau_0$, e poi per verificare che possieda la chiave privata necessaria per spendere le BTC \\ \hline
    \verb|<sig>| \verb|<pubKey>| \verb|<pubHashA>|&\verb|<pubKeyHash>| \verb|OP_EQUALVERIFY| \verb|OP_CHECKSIG|&Calcolo l'hash della chiave pubblica dell'autore di $\tau_1$ in cima allo stack e lo carico nello stack stesso \\ \hline
    \verb|<sig>| \verb|<pubKey>| \verb|<pubHashA>| \verb|<pubKeyHash>|&\verb|OP_EQUALVERIFY| \verb|OP_CHECKSIG|&Carico nello stack la chiave pubblica del destinatario di $\tau_0$\\ \hline
    \verb|<sig>| \verb|<pubKey>|&\verb|OP_CHECKSIG|&Verifico che i due elementi in cima allo stack siano uguali, ovvero che l'autore di $\tau_1$ sia anche il destinatario di $\tau_0$ \\ \hline
    \verb|true|&\verb||&Verifico la firma (ovvero il possesso da parte dell'autore di $\tau_1$ della necessaria chiave privata) e salvo il risultato nello stack \\ \hline
  \end{tabular}
  \caption{Passaggi della verifica dell'input di una transazione standard tra due utenti.}
\end{table}

L'altra transazione standard accettata dall'implementazione attuale è quella con cui si pagano le ricompense derivanti dal ritrovamento di un blocco, ovvero le transazioni \emph{coinbase} che appartengono alla tipologia \verb|pay-to-pubkey|. Non essendoci un indirizzo sorgente o una transazione in input, i valori dei campi sono più ridotti: \verb|scriptPubKey| contiene \verb|<pubKey> OP_CHECKSIG| mentre \verb|scriptSig| contiene unicamente \verb|<sig>|. Il funzionamento di questo script è descritto nella tabella \ref{table:pay-to-pubkey} e si usano le stesse convenzioni della tabella precedente.

\begin{table}
  \centering
  \label{table:pay-to-pubkey}
  \begin{tabular}{m{0.3\linewidth-2\tabcolsep} | m{\dimexpr 0.3\linewidth-2\tabcolsep} | m{\dimexpr 0.3\linewidth-2\tabcolsep}}
    \hline
    Stack&Script&Descrizione \\
    \hline
    \verb||&\verb|<sig>| \verb|<pubKey>| \verb|OP_CHECKSIG|&Unisco \verb|scriptSig| da $tau_1$ e \verb|scriptPubKey| da $tau_0$ creando lo script da analizzare  \\ \hline
    \verb|<sig>| \verb|<pubKey>|&\verb|OP_CHECKSIG|&Carico nello stack la firma dell'autore di $tau_1$ e la chiave pubblica del destinatario di $tau_0$ \\ \hline
    \verb|true|&\verb||&Verifico la firma e salvo il risultato nello stack \\ \hline
  \end{tabular}
  \caption{Passaggi della verifica di una transazione standard di generazione di BTC.}
\end{table}

Con il linguaggio di scripting è anche possibile effettuare alcune transazioni particolari non ancora considerate standard, per realizzare le quali è necessario utilizzare dei clienti appositi in grado di manipolare la struttura grezza delle transazioni. Ad esempio è possibile creare una transazione il cui output non è spendibile semplicemente inserendo all'inizio di \verb|scriptPubKey| il comando \verb|OP_RETURN|, il quale termina subito l'esecuzione dello script rendendo la transazione non utilizzabile in un futuro input in quanto il contenuto di \verb|scriptSig| diventa inutile. Esiste un esempio di tale tipo di transazione\footnote{\url{https://blockexplorer.com/tx/eb31ca1a4cbd97c2770983164d7560d2d03276ae1aee26f12d7c2c6424252f29}} che avendo l'unico output impostato a 0, lascia l'intero ammontare di 0.125 BTC come transaction fee al minatore che troverà il blocco.\\
Dal lato opposto, è anche possibile creare transazioni spendibili da chiunque\footnote{Un motivo per utilizzare in futuro questo tipo di transazioni è illustrato nel Wiki all'indirizzo \url{https://en.bitcoin.it/wiki/Fidelity_bonds}.} lasciando vuoto \verb|scriptPubKey| e inserendo \verb|OP_TRUE| in \verb|scriptSig|.\\
Un ulteriore esempio delle potenzialità dello scripting è la creazione di alcuni puzzle crittografici che devono essere risolti per poter spendere le BTC\. Esiste un esempio \footnote{\url{https://blockexplorer.com/tx/a4bfa8ab6435ae5f25dae9d89e4eb67dfa94283ca751f393c1ddc5a837bbc31b}} in cui si richiede di trovare qualche dato tale che, sottoposto due volte ad hash, risulti uguale ad un dato fornito. La tabella \ref{table:puzzle-transaction} illustra l'esecuzione di un simile script.

\begin{table}
  \centering
  \label{table:puzzle-transaction}
  \begin{tabular}{m{0.3\linewidth-2\tabcolsep} | m{\dimexpr 0.3\linewidth-2\tabcolsep} | m{\dimexpr 0.3\linewidth-2\tabcolsep}}
    \hline
    Stack&Script&Descrizione \\
    \hline
    \verb||&\verb|<data>| \verb|OP_HASH256| \verb|<given_hash>| \verb|OP_EQUAL|&Unisco \verb|scriptSig| e \verb|scriptPubKey| creando lo script da analizzare  \\ \hline
    \verb|<data>|&\verb|OP_HASH256| \verb|<given_hash>| \verb|OP_EQUAL|&Carico nello stack \verb|scriptSig| \\ \hline
    \verb|<data_hash>|&\verb|<given_hash>| \verb|OP_EQUAL|&Calcolo l'hash del dato \\ \hline
    \verb|<data_hash>| \verb|<given_hash>|&\verb|OP_EQUAL|&Carico nello stack l'hash di confronto \\ \hline
    \verb|true|&\verb||&Confronto i due hash \\ \hline
  \end{tabular}
  \caption{Passaggi della verifica di una transazione di esempio contente un puzzle.}
\end{table}

\subsection{Contratti}\label{contratti}

I contratti distribuiti sono un modo per sfruttare le potenzialità dello scripting Bitcoin per creare accordi tra persone in modo da limitare al minimo la fiducia da riporre in esse. Come per alcuni degli script precedentemente descritti, anche questi non sono considerati standard e sono anzi da considerare solo come esempi di una possibile futura evoluzione delle rete che permette nuove forme di transazioni evadibili in automatico tramite la blockchain senza necessità di intervento umano. Gli esempi sono presi dal Wiki ufficiale (\cite{bitcoin-contracts}) che a sua volta prende spunto da una dissertazione di Nick Szab\'{o} (\cite{nick-szabo}).\\

La teoria alla base dell'idea di contratto sfrutta una ulteriore proprietà latente delle transazioni Bitcoin, il \emph{time-lock}. Ogni transazione può infatti possedere un blocco temporale che ne consente la modifica e l'eliminazione entro una certa tempo (ad esempio un timestamp o l'indice di un blocco) e se impedisce lo sfruttamento dei BTC coinvolti fino a quel momento. Una transazione con time-lock che ha superato l'istante previsto diventa definitiva.
Nel caso di modifica di una transazione viene incrementato un contatore interno (uno per ogni voce di input della transazione) fino ad un valore massimo. Per cui anche nel caso in cui il time-lock non sia stato raggiunto, se tutti i contatori sono arrivati al loro valore massimo (definito come \verb|UINT_MAX|), allora la transazione viene considerata definitiva. I contatori possono essere utilizzati per generare nuove versioni di una transazione senza invalidare le firme precedenti, ad esempio nel caso in cui si voglia aggiungere un input proveniente da un diverso utente mediante l'utilizzo dei flag precedentemente accennati. Tali flag \verb|SIGHASH| definiscono il modo in cui viene creato l'hash che viene utilizzato nella fase di verifica della firma e permettono dunque di costruire transazioni multi-utente in cui ciascuno degli utenti coinvolti firma una parte della transazione senza dover (e poter) firmare o influenzare le altre parti.\\
I flag sono composti da due componenti, una modalità e un modificatore. Le modalità possibili sono 3:
\begin{description}
    \item[\PVerb{SIGHASH_ALL}:] è il valore di default, indica che tutta la transazione viene firmata ad eccezione degli script di input (che se venissero firmati anch'essi renderebbero impossibile la verifica, per cui sono sempre esclusi dalla firma). Vengono quindi incluse nella firma le altre proprietà dell'input quali gli output connessi e i contatori. È equivalente a firmare un contratto che reciti ``Sono d'accordo di impiegare danaro proveniente da questi miei fondi se tutti impiegano il danaro proveniente dai loro fondi e il totale impiegato è quello definito.''.
    \item[\PVerb{SIGHASH_NONE}:] gli output non vengono firmati e possono contenere qualsiasi cosa. Il contratto sarebbe ``Sono d'accordo di impiegare danaro proveniente da questi miei fondi fintanto che tutti impiegano il danaro proveniente dai loro fondi, ma non sono interessato a come verrà impiegata tale somma o se verrà usata nella sua totalità.''. Con questa modalità si autorizza altri utenti ad aggiornare la transazione modificando i contatori relativi ai propri input.
    \item[\PVerb{SIGHASH_SINGLE}:] come per la modalità precedente, gli input sono firmati ma non i contatori, in modo che altri utenti possano aggiornare la transazione. L'unico output che viene firmato è quello inserito nella transazione nella stessa posizione dell'input\footnote{Per chiarezza, tornando all'array degli input e degli output in formato JSON, l'output all'indice $x$ verrà firmato tramite l'input  all'indice $x$}. Il contratto equivalente è ``Accetto il contratto fintanto che il mio output sia quanto io ho stabilito, indipendentemente dagli altri firmatari.''.
\end{description}


\section{Statistiche}

Le tecniche descritte in \ref{quantificazione-della-privacy} sono state utilizzate dai ricercatori Dorit Ron e Adi Shamir del Weizmann Institute of Science di Israele per condurre alcune analisi statistiche sulla rete delle transazioni Bitcoin dalla prima transazione del 3 Gennaio 2009 al 13 Maggio 2012. I risultati da loro trovati verranno qui esposti in forma sintetica sotto forma di punti solo come nota informativa di alcune proprietà della rete, in quanto la trattazione è ormai vecchia di quasi 2 anni, durante i quali il mondo Bitcoin si è espanso esponenzialmente, minando la corrispondenza di tali statistiche con l'attuale realtà. I risultati completi sono disponibili in \cite{transazioni}. Statistiche aggiornate sono invece disponibili in tempo reale presso \cite{blockchain-info}.\\

Prima di esporre i risultati è importante specificare alcune osservazioni fatte dagli stessi osservatori. Come descritto in \ref{quantificazione-della-privacy}, il grafo delle transazioni viene chiamato \emph{grafo degli indirizzi} mentre il grafo derivato dall'unificazione degli indirizzi prende il nome di \emph{grafo delle entità}, il quale è per forza di cose un risultato approssimato e potrebbe essere vittima di due stime errate. È possibile infatti che si sia verificata una sottostima nell'assegnazione di un indirizzo ad una entità nel caso in cui non ci siano abbastanza dati (transazioni) per associare tale indirizzo ad una entità, ma è anche possibile sovrastimare il numero di entità nel caso in cui molti utenti abbiano deciso di unificare le loro attività in una unica transazione a cui ciascuno di essi contribuisce con il proprio indirizzo. Avendone parlato con gli stessi sviluppatori Bitcoin, i ricercatori hanno stabilito che gli errori di sovrastima sono estremamente ridotti, ma potrebbero essere stati commessi alcune sottostime, come già rilevato in \ref{quantificazione-della-privacy}.\\

I risultati più rilevanti della trattazione sono i seguenti:
\begin{itemize}
    \item I dati analizzati contano 3730218 indirizzi distinti, dei quali 609270 compaiono solo come destinatari di transazioni.
    \item I 3120948 indirizzi che inviano BTC sono stati aggregati in 1851544 entità distinte, che unite agli indirizzi unicamente destinatari, porta il totale delle entità a 2460814.
    \item Esiste una varianza notevole nelle statistiche a causa di una sola entità che possiede 156722 indirizzi. L'entità in questione è l'exchange Mt. Gox, responsabile all'epoca di circa il 90\% del totale degli acquisti di BTC.
    \item Dato che il bilancio medio di un indirizzo è di 2.4 BTC, e quello di una entità è 3.7 BTC, a causa degli errori sopra descritti si stima che il bilancio medio di un utente sia superiore alle 3.7 BTC.
    \item All'epoca dell'analisi esistevano 9000050 BTC, generati dai 180001 blocchi analizzati, ma sommando i bilanci degli indirizzi che hanno solo ricevuto BTC si vede che essi contengono 7019100 BTC, ovvero il 78\% del totale.
      \begin{itemize}
          \item Di queste, il 76.5\% (il 59.7\% del totale) sono \emph{vecchie BTC}, nel senso che sono state ricevute più di tre mesi prima del 13 Maggio 2012 e che non sono mai state spese fino a tale data.
          \item Alcune di queste monete potrebbero essere smarrite o abbandonate da utenti che muovevano i loro primi passi nel mondo ancora acerbo di Bitcoin, e non per forza BTC accumulate per avidità.
          \item Per precauzione Dorit e Adi hanno deciso di ignorare tutte le transazioni precedenti il 18 Luglio 2010, data di apertura di Mt. Gox, portando il numero di monete \emph{sparite} dalla rete a 1657480.
          \item Ripetendo i calcoli, il 73\% di tutte le BTC rimanenti è depositato in indirizzi dormienti, e il 70\% di tali indirizzi (il 51\% del totale) risultano vecchie.
      \end{itemize}
    \item Calcolando in base agli indirizzi invece che in base alle transazioni, risulta che il 55\% di tutte le BTC è dormiente, dato non affetto da eventuali errori di stima precedentemente descritti.
    \item Il numero totale di BTC coinvolte in transazioni fino al 13 Maggio 2012 è di 423287950 (mining escluso).
    \item Il 52\% delle entità (59\% degli indirizzi) hanno ricevuto meno di 10 BTC, e l'88\% delle entità (91\% degli indirizzi) meno di 100 BTC.
    \item 76 entità (129 indirizzi) hanno ricevuto tra le 400000 e le 800000 BTC e solo quattro entità (un indirizzo) hanno ricevuto più di 800000 BTC.
    \item Al 13 Maggio 2012 il bilancio del 97\% (98\%) di tutte le entità (indirizzi) è minore di 10 BTC. I valori cambiano a 88\% (91\%) se si considera il valore massimale mai raggiunto durante l'intera vita dell'entità (indirizzo).
    \item Solo 78 entità e 70 indirizzi hanno un bilancio finale maggiore di 10000 BTC. I numeri aumentano a 3812 entità e 3876 indirizzi se si osserva il bilancio massimale.
    \item Il 97\% (98\%) delle entità (indirizzi) hanno meno di 10 transazioni a testa, mentre 75 entità e 80 indirizzi ne hanno almeno 5000.
    \item La maggior parte delle transazioni sono piccole:
      \begin{itemize}
          \item Il 28\% delle transazioni nel grafo delle entità (47\% in quello degli indirizzi) muove meno di 0.1 BTC.
          \item Il 73\% delle transazioni nel grafo delle entità (84\% in quello degli indirizzi) muove meno di 10 BTC.
          \item Esistono solo 364 transazioni nel grafo delle entità (340 in quello degli indirizzi) che muovono più di 50000 BTC.
      \end{itemize}
    \item Mt. Gox possiede il maggior numero di indirizzi (156722) ma non il maggior numero di transazioni (477526).
    \item Una entità non meglio identificata che possede il secondo maggior numero di indirizzi (il 50\% di quelli di Mt. Gox, ovvero 246012) ha ricevuto il 31\% in più di BTC rispetto a Mt. Gox (2886650 contro 2206170).
    \item La pool Deepbit ha generato il 70\% di transazioni in più rispetto a Mt. Gox (814044 in totale), pur possendo solo due indirizzi.
\end{itemize}

Un fatto interessante riguarda le transazioni con più di 50000 BTC. Esse sono esattamente 364, la prima delle quali datata 8 Novembre 2010 e valente 90000 BTC. Tracciando il flusso delle altre 363, i ricercatori si sono resi conto che 348 sono in realtà successori della prima grande transazione di Novembre. Il grafo che descrive tale transazione mostra catene di transazioni molto lunghe, suddivisione e riunificazioni di BTC e self-loop, tutti descritti nel dettaglio in \cite{transazioni} e molto probabilmente effettuati nel tentativo (evidentemente infruttuoso) di mascherare la destinazione e la provenienza del denaro.

\section{Deflazione}\label{deflazione}

Come si è detto, Bitcoin è un sistema monetario privo di inflazione, in quanto nessuna entità centrale può emettere moneta.
Esiste però una elevatissima propensione alla deflazione\footnote{Diminuzione generale del livello dei prezzi.} la quale potrebbe avere serie conseguenze per la sicurezza dell'intera rete.
Dato il loro estremo valore in rapporto alle monete circolanti, le BTC tendono ad essere accumulate invece che spese sparendo quindi dal sistema dal punto di vista economico. Stessa fine fanno le monete per le quali è stata persa la chiave privata che permetteva loro di essere spese.
Mano a mano che le BTC spariscono dalla circolazione (per volontà o incidente), diminuisce il volume delle transazioni, quindi anche il numero di blocchi generati nell'unità di tempo e di conseguenza il numero di nuove monete introdotte in circolazione, il quale viene già dimezzato ad intervalli programmati in modo da generare in assoluto al massimo 21 milioni di BTC.
L'attrattiva della creazione di nuovi blocchi diventa quindi via via sempre più ridotta, in quanto non solo ci saranno meno BTC generate, ma anche meno tasse da parte degli utenti a causa delle minori transazioni.
Se la circolazione di BTC si riduce troppo, potrebbe verificarsi una perdita di interesse nel sistema, il che si traduce in un numero sempre minore di miners e quindi di ``verificatori'' per le transazioni, fino al punto in cui il sistema è diventato troppo debole per potersi difendere da fork nella blockchain o altri attacchi.\\
Purtroppo non esistono contromisure pratiche per il rischio di deflazione, e l'unico modo per rendere Bitcoin resistente a tale pericolo consiste nel rendere consapevoli gli utenti ed in particolare i minatori di quanto importante sia il loro contributo per la salvaguardia del loro stesso denaro.\\
Si può però imparare da Bitcoin per una futura moneta elettronica che sia resistente alla deflazione integrando un sistema di feedback per controllare il tasso di mining globale, in modo che rifornire il sistema nel caso la circolazione diminuisca.

\section{Esempi reali ???titolo da cambiare}

\subsection{Distributore automatico}

\subsection{Bancomat}

\subsection{Silkroad ??FORSE???}

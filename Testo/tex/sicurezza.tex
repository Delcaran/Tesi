\chapter{Privacy e Sicurezza}\label{privacy-e-sicurezza}

\section{Reti P2P in genere}\label{reti-p2p-in-genere}

Una rete di computer, di qualsiasi tipo essa sia, richiede
necessariamente un certo livello di sicurezza. Questo è ancora più vero
per reti pubbliche in cui chiunque può inserirsi, come le reti P2P che
si appoggiano a Internet.

Tali reti sono un bersaglio particolarmente ghiotto per gli attaccanti
proprio a causa della loro maggiore forza: il grande numero di utenti,
che equivale ad un grande numero di possibili bersagli. Ogni rete P2P
deve quindi confrontarsi con la certezza che alcuni dei suoi nodi siano
malevoli.

Data la grande varietà di reti P2P, è necessario specificare cosa
intendiamo per ``reti P2P in genere''. Il modello \cite{vulenrabilities}
a cui si farà riferimento consiste nelle seguenti componenti di base:

\begin{itemize}
\itemsep1pt\parskip0pt\parsep0pt
\item
  Uno spazio degli ID composto da b bits.
\item
  Un sistema di mappatura degli ID.
\item
  Un sistema di routing, che utilizza una chiave per inoltrare un
  messaggio alla sua destinazione. Questo include una serie di regole
  per il churm.
\end{itemize}

/// TODO: vedere se usare la documentazione del prof per descrivere i
modelli di rete P2P\ldots{}un casino

Basandoci sul modello di rete OSI, classifichiamo gli attacchi come di
basso, medio e alto livello, a seconda della vicinanza al livello di
comunicazione fisico: più il livello dell'attacco è alto, più si basa
sul software invece che sull'hardware.

Visto che le reti P2P sono costruite sulla base di altre reti, i livelli
ISO/OSI a cui siamo interessati sono il livello applicativo (per il P2P
vero e proprio) e il livello di comunicazione (per attacchi basati su
TCP e IP).

/// TODO: grafico dei livelli di attacco in confronto ad ISO/OSI (da
vulnerabilities)

\subsection{Attacchi di Basso Livello}\label{attacchi-di-basso-livello}

\subsubsection{Denial of Service (DoS)}\label{denial-of-service-dos}

Uno degli attacchi più basilari che interessa praticamente ogni tipo di
rete. Consiste nel bloccare i servizi offerti da uno specifico
bersaglio. Si tratta di un attacco estremamente comune esistente in
infinite varianti, ma nel caso di P2P la sua versione più semplice
consiste nel flood. Consiste nell'estremizzare quanto descritto nel
query flood (\textbackslash{}\TODO: collegamento a query flood):
inondare la rete di pacchetti, legittimi o creati ad-hoc, in modo da
ostacolare la normale comunicazione tra i nodi. Semplice, ma
estremamente efficace.

Una variante ancora più efficace, e resa tale proprio dalla natura
distribuita delle reti P2P, consiste nel \textbf{Distribute Denial of
Service} (DDoS). Come dice il nome, la differenza sta nel fatto che
l'attaccante non è un singolo nodo bensì un insieme di nodi. Il
vantaggio in questa tecnica, oltre che nell'immenso numero di pacchetti
che inondano la rete, sta nell'anonimato che circonda l'attaccante.
Spesso infatti i nodi che a tutti gli effetti inviano i pacchetti sono
inconsapevoli di essere parte dell'attacco e manipolati remotamente da
un attaccante. Questo aggiunge un ulteriore strato tra l'attaccante e i
nodi legittimi che tentano di impedire l'attacco.

Gli attacchi DoS e DDoS diventano tanto più probabili e quanto è vasta
la rete, non solo per il maggior numero di nodi compromettibili per un
DDoS, ma anche per le politiche aziendali attuate da molte società e
provider. Infatti, più la rete è diffusa, più è probabile che firewall
aziendali ne limitino l'accesso ai propri utenti e che ISP nazionali ne
limitino l'utilizzo. Questo costringe gli utenti a piazzarsi al di fuori
di tali reti protette per poter usufruire della rete P2P, e quindi ad
esporsi maggiormente agli attacchi.

\paragraph{Contromisure al DDoS}\label{contromisure-al-ddos}

Il primo problema consiste nell'identificare i DDoS. I sintomi di un
DDoS sono del tutto identici a quelli di un normale elevato traffico di
rete, in particolare quando i pacchetti usati sono legittimi e non
forgiati ad-hoc per l'attacco. Inoltre in un DDoS non tutti i nodi sono
stati creati appositamente per attaccare, spesso sono nodi legittimi che
vengono utilizzati dall'attaccante per far rimbalzare i suoi pacchetti
di attacco. Questi due fattori combinati rendono di fatto
\emph{impossibile} bloccare tutti gli attacchi DoS.

Esiste però una tecnica ampiamente diffusa per rendere poco pratico il
DoS, o almeno per rallentarlo in maniera drastica. Si tratta del
\textbf{pricing}, una tecnica che limita la velocità alla quale i nodi
possono fare richieste alla rete. Se un nodo deve fare richieste ad un
altro nodo (ad esempio, una query per un file), il nodo risponde con
richieste di calcoli di hash, esattamente gli stessi calcoli necessari
per creare un blocco in Bitcoin. Solo dopo aver risolto il calcolo ed
inviata la risposta, la richiesta viene presa in considerazione, tutte
le altre comunicazione inviate nel frattempo vengono scartate,
scoraggiando quindi qualsiasi richiesta invadente.

\subsubsection{Attacco Man-in-the-Middle
(MitM)}\label{attacco-man-in-the-middle-mitm}

Questo attacco consiste nell'inserimento dell'attaccante tra due nodi
della rete, in modo che tutte le comunicazioni tra i due nodi passino
attraverso l'attaccante. L'attacco è irrilevabile fin tanto che
l'attaccante rimane passivo. Una volta ottenute tutte le informazioni
che desidera, l'attacante può diventare attivo modificando i messaggi
che vengono scambiati oppure forgiandone di propri spacciandosi per uno
o entrambi dei nodi. Inoltre, dato che l'attacante può influenzare la
visione che i due nodi hanno del resto della rete, può creare false
identità per simulare messaggi legittimi.

Se questo attacco viene effettuato al livello di rete, l'attaccante è in
grado di vedere tutto ciò che passa tra i due nodi e, essendo questo
livello inferiore a quello P2P, l'attaccante non ha nessun problema nel
creare qualsiasi tipo di pacchetto P2P egli desideri.

Come il DoS ma in misura estremamente maggiore, le reti P2P sono un
bersaglio goloso per questo genere di attacchi. Questo perché è
difficile inserirsi tra due nodi in una rete normale, ma in una rete P2P
è estremamente banale: tali reti infatti non hanno nessun controllo su
come sono localizzati i nodi (\textbackslash{}\TODO: anche qui la stessa
roba del tipo di reti del libro del prof) e sono quindi
\emph{estremamente} vulnerabili al MitM. Dato che l'attaccante può
piazzarsi dove vuole all'interno della rete, gli attacchi risultano
essere molto specifici e deterministici, fino anche ad impedire ad un
determinato nodo di accedere ad un altro nodo.

\subsubsection{Contrastare il
Man-in-the-Middle}\label{contrastare-il-man-in-the-middle}

Il metodo principale per difendersi dal MitM è rendere tale attacco il
più infruttuoso possibile. In una rete senza nodi privilegiati (ad
esempio un server centrale, un supernodo o un'autorità centrale di
autenticazione), il MitM si limita a compromettere la sicurezza tra due
soli nodi nella rete, senza minacciare per nulla il resto dei nodi
\footnote{almeno fino a quando la perdita di tre nodi (due vittime e un
  attaccante) risulta tollerabile per la rete, il che è vero per tutte
  quelle reti ampiamente diffuse.}.

Molte reti (non bitcoin) sfruttano nodi privilegiati per affrontare
altre minacce o come parte integrante della loro struttura
(\textbackslash{}\TODO: vedi capitolo su gnutella e co), e anche nelle
reti maggiormente distribuite è possibile effettuare un attacco MitM su
larga scala (\emph{Eclipse}, \textbackslash{}\TODO: link a più avanti).

Il metodo più diffuso per evitare la fuoriuscita di informazioni nella
comunicazione tra due nodi è la cifratura a chiave pubblica. Con tale
sistema si garantisce l'origine del messaggio, il fatto che non è stato
alterato in alcun modo e, volendo, fornisce anche un metodo per evitare
che una terza parte non autorizzata possa leggerne il contenuto.
L'implementazione di questo semplice meccanismo rende praticamente
inutile qualsiasi tentativo di MitM tra due nodi.

\subsection{Attacchi di Medio Livello}\label{attacchi-di-medio-livello}

\subsubsection{Worms}\label{worms}

Un worm è un programma auto-replicante simile a un virus che, a
differenza di questo, è indipendente da altri programmi presenti sul
sistema. La minaccia rappresentata da un worm è decisamente
significativa per una rete P2P, in quanto si diffondono a tutti i nodi
della rete tramite vulnerabilità presenti ad un livello più basso
rispetto a quello della rete stessa. Pur non essendo legati direttamente
al P2P, i worm vengono diffusi in modo capillare da esso principalmente
a causa di come il P2P viene implementato. In molte architetture infatti
i nodi per comunicare tra loro devono avere installato lo stesso
software. Ciò significa che quando questo software ha una certa
vulnerabilità (ad esempio, un buffer overflow), tutti i nodi della rete
sono vulnerabili. Quindi mentre un worm ``normale'' deve effettuare una
scansione dell'intera rete per trovare degli host vulnerabili, un worm
P2P deve solo guardare le tabelle di routing e infettare tutti i nodi
vicini, diffondendosi esponenzialmente: in confronto ai worm normali, i
worm P2P infettano l'intera rete in modo praticamente istantaneo.

Oltre alla grande velocità di diffusione, il fatto che molte reti P2P
siano concepite per il file-sharing e abbiano quindi una grande
abbondanza di banda, consente al worm P2P di avere dimensioni maggiori e
di essere quindi capace di azioni ed attacchi molto più complicati
rispetto ad un worm normale, a volte grande solo come un pacchetto
TCP/IP. Molti nodi sono inoltre spesso computer personali di utenti
normali che utilizzano quotidianamente Internet. Worm sufficientemente
complessi sono in grado di monitorare l'attività dell'utente anche al di
fuori della rete P2P e di accedere a dati quali numero di carta di
credito, password di account, ecc., rendendo le reti P2P un bersaglio di
estremo valore.

Infine, i worm possono usare la rete come uno strumento. Si è parlato
prima parlando del DDoS di come un nodo qualsiasi possa diventare un
ignaro vettore di attacco (\textbackslash{}\TODO: link a DDoS). I worm
sono il modo in cui questo viene reso possibile: il worm contiene tutto
il codice necessario ad effettuare l'attacco, oltre al codice per
replicare se stesso negli altri nodi.

\paragraph{Contrastare i Worm}\label{contrastare-i-worm}

Il modo principale è mantenere le applicazioni sicure: senza una
vulnerabilità comune un worm non può diffondersi in modo efficace. La
sicurezza di un software dipende dal modo in cui esso è stato
programmato, ad esempio per ridurre il rischio di buffer overflow è
possibile usare linguaggi fortemente tipati.

Per ridurre invece l'efficacia di un worm si può decentralizzare il più
possibile la rete evitando di implementare nodi privilegiati e/o
utilizzare sistemi operativi \textbf{hardened}. OpenBSD dalla versione
3.8 per esempio utilizza indirizzi pseudocausali in fase di allocazione
della memoria, rendendo quindi difficile sfruttare vulenrabilità
presenti nei vari applicativi.

Ma il modo più pratico per difendersi dai worm è mantenere aperta la
rete. Ciò significa basarsi su standard diffusi ed aperti per i propri
applicativi. Rilasciare i protocolli al pubblico e distribuire il codice
dei propri software incoraggia altri sviluppatori ad una analisi
critica, il che porta alla più tempestiva scoperta di vulnerabilità ed
alla rapida creazione di patch, bug-fix e fork più robusti del software
in questione. Inoltre, con opportune licenze, ogni sviluppatore può
creare il proprio client per interfacciarsi con la rete costringendo un
attaccante a creare un worm specifico per ogni versione di ogni client,
e a mantenere tale worm aggiornato mano a mano che nuove vulnerabilità
vengono scoperte e risolte o nuovi client implementati. Con una grande
varietà di client a disposizione, non tutti i nodi saranno vulnerabili
allo stesso identico difetto presente in un altro client.

\subsection{Attacchi al livello P2P}\label{attacchi-al-livello-p2p}

\subsubsection{Comportamento scorretto}\label{comportamento-scorretto}

Non si tratta di danneggiare una rete intera o un singolo nodo, bensì di
trarre il massimo profitto offrendo minima collaborazione. Questo
comportamento viene definito genericamente \textbf{attacco razionale}.
La terminologia si basa sull'assunzione che dietro ogni nodo ci sia un
utente razionale che per istinto tenta di trarre il massimo beneficio
con il minimo sforzo. Ad esempio nell'ambito della rete BitTorrent un
utente scarica un file eliminandolo dalla condivisione non appena
completo, oppure riduce ai minimi termini la banda in upload
massimizzando quella in download: tale comportamente viene definito
significativamente \textbf{leeching}, letteralmente \emph{sanguisuga}.

Ci sono molte ragioni per cui un nodo debba comportarsi in questo modo:

\begin{itemize}
\itemsep1pt\parskip0pt\parsep0pt
\item
  Per preservare la banda in upload, spesso molto limitata dagli ISP.
\item
  Per motivi legali, in special modo dove la condivisione di contenuti
  protetti dal copyright può risultare in un'azione legale nei confronti
  del nodo che condivide. Data la natura aperta delle reti, spesso è
  molto facile risalire all'origine di un contenuto.
\item
  Per istinto: molte persone, se viene lasciata loro possibilità di
  scelta, tendono a non cooperare per il solo benificio di aiutare la
  comunità a cui appartengono, non importa quanto minimo sia il costo
  che comporta loro.
\end{itemize}

Come descritto, due sono i metodi in cui ci può implementare questo
``attacco'': riducendo le risorse a disposione della rete oppure
riducendo il contenuto condiviso.

\paragraph{Contrastare i comportamenti
scorretti}\label{contrastare-i-comportamenti-scorretti}

Ogni rete deve implementare il suo diverso meccanismo per favorire la
collaborazione tra i peer: in Bitcoin ci sono le transaction fee,
Napster utilizzava un meccanismo di reputazione che favoriva gli utenti
che condividevano di più, Samsara (una rete di backup distribuito)
consente ad un utente di utilizzare uno spazio su un altro nodo solo
pari a quello che l'utente mette a disposizione per gli altri nodi.

Un esempio encomiabile è quello di BitTorrent: il protocollo è
disinteressato al numero di file condivisi da un utente o dal contenuto
di per se, ma si interessa solamente di quante risorse vengono
condivise. Secondo il protocollo BitTorrent i file da condividere
vengono suddivisi in parti (\textbf{chunks}) di lunghezza variabile che
vengono barattati tra i nodi: più un nodo è disposto a dare, più si
vedrà restituire. In pratica, più è alta la velocità di upload, più gli
altri nodi assegneranno banda a quel nodo, aumentandone la velocità di
download.

\subsubsection{Attacco Sibilla}\label{attacco-sibilla}

Si ha questo attacco quando una singola entità malevola rappresenta un
grande numero (spesso estremamente elevato) di utenti in una rete P2P
con l'obiettivo di assumere il controllo di un segmento della rete.
L'attacco si implementa con l'attaccante che tenta di creare un grande
numero dei nodi a lui vicini. L'attacco diventa più efficace se
l'attaccante è in grado di decidere dove posizionarsi nella rete, in
quanto potrebbe aver bisogno di meno nodi per poter causare gravi danni
alla rete. Con molti nodi a propria disposizione è possibile controllare
tutti i messaggi che passano per il segmente formato dai nodi in
questione. Questo attacco è inoltre un \textbf{attacco gateway}, il che
sta ad indicare una categoria di attacchi solitamente usati come passo
preliminare per attacchi su vasta scala di altro tipo, come per esempio
l'attacco Eclipse descritto più avanti (///TODO collegamento ad attacco
Eclipse).

\paragraph{Contromisure all'attacco
Sibilla}\label{contromisure-allattacco-sibilla}

La natura aperta delle reti P2P gioca a favore di questo tipo di
attacco: senza un'autorità centrale è impossibile fermare completamente
un attacco Sibilla. Il meglio che si può fare è renderlo impraticabile.

Per rallentare un attacco si può usare lo stesso metodo usato con gli
attacchi DoS: il pricing (///TODO: collegamento a pricing in DoS). Il
grande numero di calcoli richiesti per unire molti nodi alla rete può
richiedere all'attaccante più tempo di quanto sia disposto ad
impiegarne. Se inoltre la rete implementa una sorta di tempo massimo
durante il quale un nodo mantiene un certo identificativo, tutti gli
attacchi hanno un tempo massimo entro il quale devono essere portati a
termine, dopo di che i nodi creati dovranno ricollegarsi alla rete e
quindi si dovranno ripetere tutte le pratiche del pricing.

\subsubsection{Attacco Eclipse}\label{attacco-eclipse}

L'obiettivo di questo attacco è di separare la rete in due o più
partizioni. Quando l'attacco ha successo, tutte le comunicazioni tra le
due partizioni devono passare attraverso un singolo nodo malevolo. In
pratica questo risulta in un attacco Man-in-the-Middle su vasta scala
eseguito a livello applicativo e non a livello di rete, con tutte le
potenzialità del MitM normale. Per eseguire l'attacco bisogna piazzare i
propri nodi in punti di routing strategici che esistono tra le due
partizioni che si vogliono creare. Un attacco Eclipse di successo può
distruggere qualsiasi rete P2P, soprattutto quelle che non si curano
molto di mantenere tabelle di routing efficienti, perché i nodi finti
possono essere posizionati in modo da riempire i vuoti nelle tabelle di
routing di ogni altro nodo.

\paragraph{Contromisure ad Eclipse}\label{contromisure-ad-eclipse}

Data la similarità di Eclipse con il MitM, le contromisure sono anche
similari: cifratura a chiave pubblica. Tuttavia, sebbene MitM non sia
una minaccia per la rete, le dimensioni di Eclipse lo rendono
estremamente pericoloso anche in caso di cifratura a chiave pubblica. Se
i messaggi illeciti indirizzati ai nodi legittimi vengono bloccati, le
due partizioni create da Eclipse rimangono di fatto isolate. Con
abbastanza nodi piazzati in punti strategici della rete, è possibile
creare quante divisioni si desidera, riducendo quindi le dimensioni
della rete.

Come per l'attacco Sibilla, è importante impedire ad un attaccante di
decidere dove piazzare i suoi nodi. Questo significa che sarà richiesto
un elevato numero di nodi per avere una speranza di ottenere sufficiente
controllo per poter creare una partizione. Per cui è importante notare
che con un attacco Sibilla sufficiente grande è \emph{sempre} possibile
eseguire un attacco Eclipse.

\subsection{Contromisure generali}\label{contromisure-generali}

Si è quindi visto come, oltre alle caratteristiche di tolleranza ai
guasti e grande scalabilità che ne rappresentano la base del design, le
reti P2P devono anche essere progettate per difendersi dagli attacchi.
Solo in questo modo una rete potrà consentire ad ogni nodo di collegarsi
e realizzare in pieno il concetto di collaborazione che sta alla base.

In particolare i progettisti di reti P2P dovrebbero implementare le
seguenti caratteristiche:

\begin{itemize}
\itemsep1pt\parskip0pt\parsep0pt
\item
  L'impossibilità per un nodo di decidere in che punto della rete
  piazzarsi.
\item
  Un limite per la velocità di churm per i nuovi nodi.
\item
  Limitare la velocità di scambio di messaggi tra i nodi, ad esemppio
  con un princing.
\item
  Utilizzare la crittografia a chiave pubblica per garantire solo
  messaggi legittimi tra i nodi.
\item
  Usare ed implementare solo standard aperti, per diversificare il
  software a disposizione degli utenti ed irrobustire quelli esistenti.
\end{itemize}

Se queste caratteristiche vengono implementate, allora tutti gli
attacchi fin qui descritti perdono gran parte della loro efficacia,
ripagando con la sicurezza il costo che richiede la loro realizzazione.

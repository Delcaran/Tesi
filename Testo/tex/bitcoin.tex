\chapter{Bitcoin: moneta elettronica
decentralizzata}\label{bitcoin-moneta-elettronica-decentralizzata}

Tutte le reti P2P finora descritte sono in circolazione da molti anni,
hanno una base di utenti che conta milioni di peer divisi tra utenti
reali e server automatizzati, contano migliaia di forum di supporto e
scambiano quotidianamente una immensa fetta del traffico totale della
rete Internet (tanto che molti provider tendono a limitarne quanto più
possibile l'utilizzo, soprattutto nelle fasce orarie di maggior
traffico). Sono però tutte reti dedicate al file-sharing. Bitcoin no. O
almeno, non proprio, come vedremo.

Bitcoin è una rete P2P (intesa per tutti e tre i livelli descritti in
precedenza) che mira a creare un sistema di valuta digitale privo di
controllo centrale, con pagamenti effettuati direttamente tra gli utenti
senza l'intervento di terzi. È stata ideata e realizzata in origine da
un anonimo noto con il nome di \textbf{Satoshi Nakamoto} \cite{bitcoin}
e si è in poco tempo evoluta in modo esponenziale fino a catturare di
recente l'attenzione dei media internazionali, delle banche mondiali e,
per alcuni suoi utilizzi illeciti, da FBI ed NSA. Ma vediamo di cosa si
tratta.

\section{Moneta Elettronica}\label{moneta-elettronica}

\textbf{Bitcoin} è il nome dato alla rete, al client originale, al
protocollo di comunicazione e alla moneta utilizzata per le transazioni
all'interno della rete.

Le monete (d'ora in avanti \textbf{btc}) posso essere ottenute
``gratuitamente'' dopo aver impegnato la propria CPU o GPU in alcuni
calcoli di crittografia (operazione chiamata \textbf{mining} e discussa
più avanti), oppure acquistate da altri utenti della rete tramite una
valuta reale \footnote{Ad esempio tramite il sito Internet Mt. Gox
  \cite{mtgox}.}.

Entrambi questi metodi sono fondamentali nell'ecosistema Bitcoin:

\begin{itemize}
\item
  Grazie al mining, la valuta Bitcoin è ininflazionabile. Gli algoritmi
  che permettono la creazione di bitcoin ``dal nulla'' sono progettati
  in modo da limitare il numero massimo di bitcoin creabili ad un velore
  tendente le 21 milioni di unità. A Gennaio 2013 sono state generate
  circa 10 milioni di bitcoin e si stima che nel 2017 si raggiungerà la
  quota di 15 milioni. La quota massima di moneta circolante e l'assenza
  di istituti centrali in grado di creare nuove monete rendono
  l'economia bitcoin invulnerabile all'inflazione che colpisce le
  economie reali. Il sistema è ispirato a quella che era l'economia del
  Dollaro prima della istituzionalizzazione della Federal Reserve come
  banca federale: il valore del Dollaro era legato al valore corrente
  dell'oro, il quale esiste in quantità limitata ed è ottenibile solo
  attraverso il lavoro dei minatori.
\item
  La compravendita di bitcoin è invece simile alle compravendite di
  azioni effettuate nelle borse di tutte il mondo. Esistono infatti
  alcuni luoghi dedicati (ad esempio il sito internet \emph{Mt. GOX})
  che fungono da stock exchange offrendo agli utenti la possibilità di
  mettere in vendita o di acquistare bitcoin al prezzo che preferiscono.
  Sono delle vere e proprie borse che trattano unicamente bitcoin invece
  che molti titoli di aziende diverse, calcolano un valore di scambio
  medio basato sulle ultime transazioni portate a termine ma lasciano
  libero l'utente di scegliere a quanto vendere o comprare bitcoin, con
  prezzo medio che si adegua di conseguenza. \%fixme sistemare sta roba
  \%TODO: verificare se rapporto bitcoin/euro e bitcoin/dollaro sono
  legati alle transazioni in quella valuta (come credo che sia)
\end{itemize}

Come per le monete in valuta reale e le azioni borsistiche, anche le
bitcoin vengono ``tenute'' in portafogli. Come vedremo, il termine
\emph{tenute} è usato impropriamente, ma questa è l'apparenza dal punto
di vista dell'utente, per cui a tale apparenza al momento ci atterremo.

\subsection{Portafogli e indirizzi}\label{portafogli-e-indirizzi}

La prima volta che un nuovo utente avvia il suo client Bitcoin fresco di
installazione, si vede assegnato un portafoglio contenente un indirizzo
e una copia di chiavi di cifratura simmetrica. L'indirizzo è
semplicemente una stringa di 33 caratteri alfanumerici che inizia con un
1 o con un 3, generata in modo casuale dalle chiavi create per l'utente.
Gli indirizzi rappresentano il punto di uscita e/o il punto di ingresso
per tutti i movimenti che coinvolgono bitcoin. Questo significa che
nelle transazioni bitcoin compariranno unicamente questi indirizzi,
rendendo di fatto \textbf{anonimi} tutti i movimenti di bitcoin (meno
quelli che riguardano l'acquisto di bitcoin tramite moneta reale), in un
modo del tutto equivalente a quello dei conti in Svizzera. Non esiste
quindi nessuna correlazione diretta e ovvia tra un utente e il suo
indirizzo. Essendo le chiavi associate al portafogli, e gli indirizzi
generati dalle chiavi, viene di conseguenza pensare (giustamente) che un
portafogli possa contenere più indirizzi: basta infatti usare le chiavi
per generare un nuovo indirizzo e il gioco è fatto. Il numero di
indirizzi esistenti è virtualmente infinito.

I portafogli sono solitamente legati al software che li crea, basta
quindi cambiare software per poter creare un nuovo portafogli e una
nuova coppia di chiavi (oppure installare un software in grado di
gestire più portafogli). In alternativa è possibile ``aprire un conto''
presso numeri siti che offrono questa funzionalità, quale ad esempio
blockchain.info. In questo caso però bisogna fare i conti con la
sicurezza del sito in questione \%FIXME link alla sezione sicurezza

\subsection{Transazioni}\label{transazioni}

Le transazioni rappresentano il nucleo fondamentale di bitcoin. Esse
sono il metodo con cui ci si assicura che un indirizzo contenga
esattamente quel numero di bitcoin, che una bitcoin non venga spesa più
volte e che quella bitcoin appartiene a quello specifico indirizzo. Le
transazioni si basano su meccanismi di crittografia a chiave pubblica,
rendendo quindi obsoleto il coinvolgimento di terze parti nella
transazione. Se infatti nelle normali compravendite online, volenti o
nolenti si è costretti a fidarsi di terze parti che garantiscono per il
buon esito dell'operazione (istituti di credito, compagnie di carte di
credito, siti come Paypal, ecc), qui gli utenti hanno direttamente la
prova crittografica senza aver quindi necessita di fidarsi di qualcuno.

Satoshi Nakamoto descrive la sua moneta elettronica come una serie di
firme digitali. Il trasferimento di moneta da un utente all'altro
avviene infatti applicando la firma digitale dell'acquirente ad un hash
di una precedente transazione e della chiave pubblica del venditore, e
aggiungendo ciò alla fine della moneta.

Riassumendo schematicamente:

\begin{verbatim}
hash = hash(previous_transaction, vendor_public_key)
transaction = sign(hash, private_key)
\end{verbatim}

La moneta diventa quindi non una unità atomica, ma il risultato di una
serie di transazioni che coinvolgono firme digitali e verifiche che deve
essere calcolato dinamicamente. La serie di tutte le transazioni mai
effettuate viene raccolta in una sequenza denominata \textbf{blockchain}
\%TODO verificare termine blockchain

\%TODO immagine pag 2 in alto di Nakamoto

Questa implementazione però non garantisce che l'acquirente non abbia
già effettuato una transazione con questa moneta, ovvero che stia
spendendo una moneta già spesa in precedenza.

L'unico modo per garantire ciò senza utilizzare una terza parte di cui
fidarsi, è tenere conto di \textbf{tutte} le transazioni. Questo vuol
dire che tutte le transazioni devono essere annunciate ad un pubblico in
grado di mettersi d'accordo sull'effettivo ordine temporale in cui sono
state effettuate. Il venditore deve avere quindi la prova che, nel
momento in cui riceve la transazione, la maggioranza dei nodi è
d'accordo che quella è la prima transazione ricevuta.

La soluzione consiste nell'utilizzo di un \textbf{timestamp server}. Un
timestamp server funziona calcolando l'hash di un blocco di oggetti di
cui si vuole realizzare il timestamp e rendendo tale hash pubblico. Il
timestamp dimostra inequivocabilmente che gli oggetti esistevano al
momento dell'hashing. Ogni timestamp include anche il precedente
timestamp nell'hash, formando quindi una catena in cui ogni timestamp
rinforza \footnote{leggasi: rende più difficili da modificare.} quelli
precedenti.

\%TODO: grafico dei timestamp a pag 2 di Nakamoto

Ora il problema consiste nell'implementare questo server di timestamp in
modo distribuito, come è appunto la rete Bitcoin. Per prima cosa bisogna
trovare un sistema per cui effettuare il timestamp è un'operazione
difficoltosa (computazionalmente parlando), ma verificare che il
timestamp sia corretto deve essere immediato. Basandosi sul lavoro di
Adam Back (\cite{hashcash}), Nakamoto ha deciso che la difficoltà
dell'operazione deve essere trovare un valore che, una volta sottoposto
ad hashing (ad esempio con SHA-256), il risultato sia un hash che
comincia con uno specifico numero di bit pari a zero: la difficoltà del
lavoro è esponenziale al numero di bit zero richiesti, ma è facilmente
verificabile con un singolo hash. L'implementazione per bitcoin consiste
quindi nella creazione di un blocco di dati di cui calcolare l'hash che
contiene le transazioni interessate, l'hash precedente e un valore
chiamato \textbf{nonce} da incrementare fino a quando l'hash non avrà le
caratteristiche richieste. Modificare una transazione comporta
modificare un blocco, e quindi ripetere tutto il lavoro di calcolo della
nonce. Inoltre, se a questo blocco è già stato incatenato uno più
blocchi successivi, anche tali blocchi andranno ricalcolati in sequenza,
rendendo il lavoro estremamente gravoso.

\%TODO immagine sequenza di blocchi pag 3 nakamoto

Con la prova di lavoro si risolve anche il problema di cosa significa
che la maggioranza deve accettare un timestamp. Con l'hash infatti si
realizza una sorta di sistema one-CPU-one-vote, e la ``decisione della
maggioranza'' è rappresentata dalla più lunga sequenza di timestamp, che
è la sequenza per la quale è stata impiegata la maggior parte di lavoro
computazionale. Ciò significa che se la maggior parte della forza-CPU è
controllata da peer onesti (cioè che non hanno nessuna intenzione di
modificare una transazione effettuata), un nodo disonesto che volesse
modificare una transazione non solo dovrebbe rifare tutti i calcoli per
il blocco della transazione e per tutti i blocchi successivi, ma avendo
minor potenza di CPU a disposizione rispetto ai nodi onesti, verrebbe
rapidamente soverchiato dal numero di calcoli da fare, in quando il
numero di blocchi da ricalcolare sarebbe sempre superiori a quelli da
lui già ricalcolati.

Si capisce subito che è nella rete bitcoin (e nelle reti P2P in
generale) è importante che le risorse (potenza di calcolo in questo
caso) siano equamente distribuite tra i peer, in modo da evitare che un
solo nodo o un solo gruppo di nodi controlli l'intera rete.

Per far fronte alle differenti configurazioni hardware degli utenti,
alla sempre crescente capacità di calcolo di CPU e GPU e anche ai
potenzialmente mutevoli interessi dei nodi, la difficoltà della prova di
lavoro (ovvero il numero di bit zero) è determinata da una media
calcolata sul numero medio di blocchi generati ogni ora. Se vengono
generati troppi blocchi, vuol dire che la difficoltà è troppo bassa e
viene subito aumentata.

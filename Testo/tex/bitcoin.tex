\chapter{Bitcoin: moneta elettronica
decentralizzata}\label{bitcoin-moneta-elettronica-decentralizzata}

Tutte le reti P2P finora descritte sono in circolazione da molti anni,
hanno una base di utenti che conta milioni di peer divisi tra utenti
reali e server automatizzati, contano migliaia di forum di supporto e
scambiano quotidianamente una immensa fetta del traffico totale della
rete Internet (tanto che molti provider tendono a limitarne quanto più
possibile l'utilizzo, soprattutto nelle fasce orarie di maggior
traffico). Sono però tutte reti dedicate al file-sharing. Bitcoin no. O
almeno, non proprio\ldots{}

Bitcoin è una rete P2P (intesa per tutti e tre i livelli descritti in
precedenza) che mira a creare un sistema di valuta digitale priva di
controllo centrale, con pagamenti effettuati direttamente tra gli utenti
senza l'intervento di terzi. È stata ideata e realizzata in origine da
un anonimo noto con il nome di \textbf{Satoshi Nakamoto} (@bitcoin)

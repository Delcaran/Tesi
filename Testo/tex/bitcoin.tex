\chapter{Bitcoin: moneta elettronica
decentralizzata}\label{bitcoin-moneta-elettronica-decentralizzata}

Tutte le reti P2P finora descritte sono in circolazione da molti anni,
hanno una base di utenti che conta milioni di peer divisi tra utenti
reali e server automatizzati, contano migliaia di forum di supporto e
scambiano quotidianamente una immensa fetta del traffico totale della
rete Internet (tanto che molti provider tendono a limitarne quanto più
possibile l'utilizzo, soprattutto nelle fasce orarie di maggior
traffico). Sono però tutte reti dedicate al file-sharing. Bitcoin no. O
almeno, non proprio, come vedremo.

Bitcoin è una rete P2P (intesa per tutti e tre i livelli descritti in
precedenza) che mira a creare un sistema di valuta digitale privo di
controllo centrale, con pagamenti effettuati direttamente tra gli utenti
senza l'intervento di terzi. È stata ideata e realizzata in origine da
un anonimo noto con il nome di \textbf{Satoshi Nakamoto} \cite{bitcoin}
e si è in poco tempo evoluta in modo esponenziale fino a catturare di
recente l'attenzione dei media internazionali, delle banche mondiali e,
per alcuni suoi utilizzi illeciti, da FBI ed NSA. Ma vediamo di cosa si
tratta.

\section{Moneta Elettronica}\label{moneta-elettronica}

\textbf{Bitcoin} è il nome dato alla rete, al client originale, al
protocollo di comunicazione e alla moneta utilizzata per le transazioni
all'interno della rete.

Le monete (d'ora in avanti \textbf{btc}) posso essere ottenute
``gratuitamente'' dopo aver impegnato la propria CPU o GPU in alcuni
calcoli di crittografia (operazione chiamata \textbf{mining} e discussa
più avanti), oppure acquistate da altri utenti della rete tramite una
valuta reale \footnote{Ad esempio tramite il sito Internet Mt. Gox
  \cite{mtgox}.}.

Entrambi questi metodi sono fondamentali nell'ecosistema Bitcoin:

\begin{itemize}
\item
  Il mining è una diretta conseguenza della partecipazione di un nodo
  alla rete (vedi più avanti NETWORK) ed è l'unico metodo con cui
  vengono create e messe in circolazioni nuove \textbf{btc}. Per ogni
  operazione di mining avvenuta con successo si riceve una quantità
  fissa di btc che viene dimezzata nel tempo, limitando il numero
  massimo di bitcoin in circolazione a circa 21 milioni. A Gennaio 2013
  è stata generata circa la metà delle monete totali e si stima di
  arrivare a 3/4 nel 2017. La quota massima di moneta circolante e
  l'assenza di istituti centrali in grado di creare nuove monete rendono
  l'economia bitcoin invulnerabile all'inflazione che colpisce le
  economie reali. Il sistema è ispirato a quella che era l'economia del
  Dollaro prima della istituzionalizzazione della Federal Reserve come
  banca federale: il valore del Dollaro era legato al valore corrente
  dell'oro, il quale esiste in quantità limitata ed è ottenibile solo
  attraverso il lavoro dei minatori.
\item
  La compravendita di bitcoin è invece simile alle compravendite di
  azioni effettuate nelle borse di tutte il mondo. Esistono infatti
  alcuni luoghi dedicati (ad esempio il sito internet \emph{Mt. GOX})
  che fungono da stock exchange offrendo agli utenti la possibilità di
  mettere in vendita o di acquistare bitcoin al prezzo che preferiscono.
  Sono delle vere e proprie borse che trattano unicamente bitcoin invece
  che molti titoli di aziende diverse, calcolano un valore di scambio
  medio basato sulle ultime transazioni portate a termine ma lasciano
  libero l'utente di scegliere a quanto vendere o comprare bitcoin, con
  prezzo medio che si adegua di conseguenza. \%FIXME: sistemare sta roba
  \%TODO: verificare se rapporto bitcoin/euro e bitcoin/dollaro sono
  legati alle transazioni in quella valuta (come credo che sia)
\end{itemize}

Come per le monete in valuta reale e le azioni borsistiche, anche le
bitcoin vengono ``tenute'' in portafogli. Come vedremo, il termine
\emph{tenute} è usato impropriamente, ma questa è l'apparenza dal punto
di vista dell'utente, per cui a tale apparenza al momento ci atterremo.

\section{Portafogli e indirizzi}\label{portafogli-e-indirizzi}

La prima volta che un nuovo utente avvia il suo client Bitcoin fresco di
installazione, si vede assegnato un portafoglio contenente un indirizzo
e una copia di chiavi di cifratura simmetrica. L'indirizzo è
semplicemente una stringa di 33 caratteri alfanumerici che inizia con un
1 o con un 3, generata in modo casuale dalle chiavi create per l'utente.
Gli indirizzi rappresentano il punto di uscita e/o il punto di ingresso
per tutti i movimenti che coinvolgono bitcoin. Questo significa che
nelle transazioni bitcoin compariranno unicamente questi indirizzi,
rendendo di fatto \textbf{anonimi} tutti i movimenti di bitcoin (meno
quelli che riguardano l'acquisto di bitcoin tramite moneta reale), in un
modo del tutto equivalente a quello dei conti in Svizzera. Non esiste
quindi nessuna correlazione diretta e ovvia tra un utente e il suo
indirizzo. Essendo le chiavi associate al portafogli, e gli indirizzi
generati dalle chiavi, viene di conseguenza pensare (giustamente) che un
portafogli possa contenere più indirizzi: basta infatti usare le chiavi
per generare un nuovo indirizzo e il gioco è fatto. Il numero di
indirizzi esistenti è virtualmente infinito.

I portafogli sono solitamente legati al software che li crea, basta
quindi cambiare software per poter creare un nuovo portafogli e una
nuova coppia di chiavi (oppure installare un software in grado di
gestire più portafogli). In alternativa è possibile ``aprire un conto''
presso numeri siti che offrono questa funzionalità, quale ad esempio
blockchain.info. In questo caso però bisogna fare i conti con la
sicurezza del sito in questione \%TODO: link alla sezione sicurezza

\section{Transazioni}\label{transazioni}

Le transazioni rappresentano il nucleo fondamentale di bitcoin. Esse
sono il metodo con cui ci si assicura che un indirizzo contenga
esattamente quel numero di bitcoin, che una bitcoin non venga spesa più
volte e che quella bitcoin appartiene a quello specifico indirizzo. Le
transazioni si basano su meccanismi di crittografia a chiave pubblica,
rendendo quindi obsoleto il coinvolgimento di terze parti nella
transazione. Se infatti nelle normali compravendite online, volenti o
nolenti si è costretti a fidarsi di terze parti che garantiscono per il
buon esito dell'operazione (istituti di credito, compagnie di carte di
credito, siti come Paypal, ecc), qui gli utenti hanno direttamente la
prova crittografica senza aver quindi necessita di fidarsi di qualcuno.

Satoshi Nakamoto descrive la sua moneta elettronica come una serie di
firme digitali. Il trasferimento di moneta da un utente all'altro
avviene infatti applicando la firma digitale dell'acquirente ad un hash
di una precedente transazione e della chiave pubblica del venditore, e
aggiungendo ciò alla fine della moneta.

Riassumendo schematicamente:

\begin{verbatim}
hash = hash(previous_transaction, vendor_public_key)
transaction = sign(hash, private_key)
\end{verbatim}

La moneta diventa quindi non una unità atomica, ma il risultato di una
serie di transazioni che coinvolgono firme digitali e verifiche che deve
essere calcolato dinamicamente. La serie di tutte le transazioni mai
effettuate viene raccolta in una sequenza denominata \textbf{blockchain}
\%TODO: verificare termine blockchain

\%TODO: immagine pag 2 in alto di Nakamoto

Questa implementazione però non garantisce che l'acquirente non abbia
già effettuato una transazione con questa moneta, ovvero che stia
spendendo una moneta già spesa in precedenza.

L'unico modo per garantire ciò senza utilizzare una terza parte di cui
fidarsi, è tenere conto di \textbf{tutte} le transazioni. Questo vuol
dire che tutte le transazioni devono essere annunciate ad un pubblico in
grado di mettersi d'accordo sull'effettivo ordine temporale in cui sono
state effettuate. Il venditore deve avere quindi la prova che, nel
momento in cui riceve la transazione, la maggioranza dei nodi è
d'accordo che quella è la prima transazione ricevuta.i

\section{Timestamp e Proof-of-Work}\label{timestamp-e-proof-of-work}

La soluzione consiste nell'utilizzo di un \textbf{timestamp server}. Un
timestamp server funziona calcolando l'hash di un blocco di oggetti di
cui si vuole realizzare il timestamp e rendendo tale hash pubblico. Il
timestamp dimostra inequivocabilmente che gli oggetti esistevano al
momento dell'hashing. Ogni timestamp include anche il precedente
timestamp nell'hash, formando quindi una catena in cui ogni timestamp
rinforza \footnote{leggasi: rende più difficili da modificare.} quelli
precedenti.

\%TODO: grafico dei timestamp a pag 2 di Nakamoto

Ora il problema consiste nell'implementare questo server di timestamp in
modo distribuito, come è appunto la rete Bitcoin. Per prima cosa bisogna
trovare un sistema per cui effettuare il timestamp è un'operazione
difficoltosa (computazionalmente parlando), ma verificare che il
timestamp sia corretto deve essere immediato. Basandosi sul lavoro di
Adam Back (\cite{hashcash}), Nakamoto ha deciso che la difficoltà
dell'operazione deve essere trovare un valore che, una volta sottoposto
ad hashing (ad esempio con SHA-256), il risultato sia un hash che
comincia con uno specifico numero di bit pari a zero: la difficoltà del
lavoro è esponenziale al numero di bit zero richiesti, ma è facilmente
verificabile con un singolo hash. L'implementazione per bitcoin consiste
quindi nella creazione di un blocco di dati di cui calcolare l'hash che
contiene le transazioni interessate, l'hash precedente e un valore
chiamato \textbf{nonce} da incrementare fino a quando l'hash non avrà le
caratteristiche richieste. Modificare una transazione comporta
modificare un blocco, e quindi ripetere tutto il lavoro di calcolo della
nonce. Inoltre, se a questo blocco è già stato incatenato uno più
blocchi successivi, anche tali blocchi andranno ricalcolati in sequenza,
rendendo il lavoro estremamente gravoso.

\%TODO: immagine sequenza di blocchi pag 3 nakamoto

Con la prova di lavoro si risolve anche il problema di cosa significa
che la maggioranza deve accettare un timestamp. Con l'hash infatti si
realizza una sorta di sistema one-CPU-one-vote, e la ``decisione della
maggioranza'' è rappresentata dalla più lunga sequenza di timestamp, che
è la sequenza per la quale è stata impiegata la maggior parte di lavoro
computazionale. Ciò significa che se la maggior parte della forza-CPU è
controllata da peer onesti (cioè che non hanno nessuna intenzione di
modificare una transazione effettuata), un nodo disonesto che volesse
modificare una transazione non solo dovrebbe rifare tutti i calcoli per
il blocco della transazione e per tutti i blocchi successivi, ma avendo
minor potenza di CPU a disposizione rispetto ai nodi onesti, verrebbe
rapidamente soverchiato dal numero di calcoli da fare, in quando il
numero di blocchi da ricalcolare sarebbe sempre superiori a quelli da
lui già ricalcolati.

Si capisce subito che è nella rete bitcoin (e nelle reti P2P in
generale) è importante che le risorse (potenza di calcolo in questo
caso) siano equamente distribuite tra i peer, in modo da evitare che un
solo nodo o un solo gruppo di nodi controlli l'intera rete.

Per far fronte alle differenti configurazioni hardware degli utenti,
alla sempre crescente capacità di calcolo di CPU e GPU e anche ai
potenzialmente mutevoli interessi dei nodi, la difficoltà della prova di
lavoro (ovvero il numero di bit zero) è determinata da una media
calcolata sul numero medio di blocchi generati ogni ora. Se vengono
generati troppi blocchi, vuol dire che la difficoltà è troppo bassa e
viene subito aumentata.

\section{Network}\label{network}

A questo punto abbiamo una prima approssimazione di come funziona la
rete bitcoin:

\begin{enumerate}
\def\labelenumi{\arabic{enumi}.}
\itemsep1pt\parskip0pt\parsep0pt
\item
  Le nuove transazioni sono inviate a tutti i nodi.
\item
  Ogni nodo raccoglie le transazioni che riceve in un blocco.
\item
  Per ogni blocco, ogni nodo cerca di calcolare una proof-of-work.
\item
  Una volta trovata la prova, invia il blocco a tutti i nodi.
\item
  I nodi accettano il nuovo blocco se e solo se tutte le transazioni in
  esso sono valide (si verifica calcolando l'hash delle transazioni e
  confrontandole con l'ultimo blocco accettato) e non già spese in
  precedenza.
\item
  Il nodo esprime la sua accettazione del blocco appena arrivato
  mettendosi al lavoro per crearne uno nuovo, usando l'hash del nodo
  accettato.
\end{enumerate}

I nodi considerano la catena più lunga quella corretta e lavoreranno
sempre in modo da prolungarla. Esiste la possibilità che uno stesso nodo
riceva due versioni diverse dello stesso blocco in contemporanea. In
questo caso, lavoreranno sul prima blocco ricevuto, ma manterrano una
copia anche dell'altro nel caso in cui si rivelasse appartenente alla
catena più lunga. La verifica viene fatta non appena viene trovata la
nuova proof-of-work e una delle due catene si allunga: a questo punto si
individua il blocco da mantenere in base all'hash contenuto nel blocco
appena arrivato, gli altri blocchi vengono scartati e si continua il
procedimento.

Quando si inviano in broadcast le nuove transazioni, non è necessario
che esse raggiungano tutti i nodi: fintanto che raggiungono quanti più
nodi possibile, verranno velocemente inglobate in un blocco. I blocchi
invece devono essere ricevuti da tutti i nodi, per questo se un nodo
riceve un blocco e si accorge (tramite hash) che il blocco precedente
gli manca, ne richiederà immediatamente una copia ad un altro nodo, e
ripeterà la verifica fino ad ottenere la catena integrale.

Per convenzione, la prima transazione di un blocco è una transazione
speciale che crea una nuova moneta e la assegna al creatore del blocco.
In pratica il primo nodo che riesce a trovare la proof-work di un nuovo
blocco riceve un premio in bitcoin. Tale premio serve ad incentivare i
nodi a mantenere attiva la rete ed inoltre permette la messa in
circolazione di nuove monete, senza che sia necessaria una zecca
centrale. Questo profitto derivante dal calcolo della proof-of-work
viene denominato \textbf{mining} e ha spinto molti utenti ad acquistare
hardware dedicato sempre più performante in modo da creare per primi il
nuovo blocco e ottenere il relativo premio, inizialmente ammontante a
50btc ma ora dimezzato a 25btc). \%TODO: collegamento al mining

Visto che il numero massimo di btc è stato determinato a priori ed è
invariabile, per incentivare i nodi anche quando il mining risulterà
inutile, sono state introdotte delle vere e proprie tasse di
transazione, le \textbf{transaction fees}. Una transaction fee non ha un
valore fisso, ma viene decisa da chi effettua la transazione e può anche
essere nulla: viene registrata come una differenza tra il valore di
input e il valore di output della transazione, con quest'ultimo valore
inferiore del primo (come in una tassa, l'acquirente spende più
dell'importo effettivo). Tale differenza verrà trasferita nella
transazione dedicata all'incentivo al momento della creazione del nuovo
blocco, sempre ammesso che l'utente che ha creato il blocco voglia
riceve queste btc.

Oltre ad invogliare un nodo a rimanere attivo, gli incentivi
incoraggiano i nodi a rimanere onesti: infatti se un attaccante avido
riuscisse ad accumulare abbastanza potenza di calcolo da surclassare
quella di tutti gli altri nodi, potrebbe scegliere se truffare gli altri
nodi ritirando i suoi pagamenti precedenti oppure usarla per accumulare
nuove monete con gli incentivi. Un attaccante si vede quindi
incoraggiato a mettere la sua potenza di calcolo a favore del sistema
facendogli guadagnare più btc di tutti gli altri nodi messi insieme,
invece di usare la stessa potenza per minare le basi dello stesso
sistema in cui egli stessi investe i propri soldi.

\section{Spazio \%FIXME: titolo
orribile}\label{spazio-fixme-titolo-orribile}

Il sito Blockchain.info \cite{blockchain-info} offre vari servizi agli
utenti bitcoin, tra i quali spiccano un portafogli online, un sistema di
navigazione dell'intera catena dei blocchi e dettagliate statistiche
sull'intera rete. Da tale sito si vede come in media, in 24 ore vengono
effettuate 56700 transazioni archiviate in 220 blocchi, il che vuol dire
un blocco ogni 6.55 minuti. Se ogni nodo dovesse mantenere ogni singola
transazione, lo spazio di memoria occupato renderebbe la rete bitcoin
non così appetibile per l'utente medio. Risulta necessario minimizzare
la quantità di memoria necessaria a mantenere la blockchain senza
compromettere la sicurezza.

Si è detto infatti che l'hash di un blocco viene calcolato a partire
dall'hash del blocco precedente, da una nonce e dalle transazioni
contenute nel blocco. Ma invece che memorizzare le intere transazioni,
esse vengono inserite come foglie di un albero di Merkle \cite{merkle},
una struttura dati in cui ogni nodo non foglia è l'hash di tutti i suoi
figli, in modo che nella radice di tale albero ci sia un solo hash che
riassume in se tutte le transazioni. È pertanto possibile calcolare
l'hash di un blocco basandosi unicamente sull'hash del blocco
precedente, sulla nonce e sulla radice dell'albero di Merkle, ovvero
sull'hash riassuntivo delle transazioni, rendendo quindi possibile
eliminare alcune transazioni dalla blockchain per risparmiare un poco di
spazio.

\%FIXME: grafico pag 4 satoshi

Con questa struttura, il \textbf{block header} viene ad occupare
esattamente 80 Bytes (vedi sezione tecnica per i dettagli \%TODO:
sezione tecnica), il che significa circa 6.2 MB all'anno (Satoshi
Nakamoto con una stima di un blocco ogni 10 minuti aveva previsto 4.2 MB
annui). Una quantità decisamente ridotta che rende la blockchain
tollerabile su ogni computer degli ultimi 7 anni.

È possibile quindi verificare i pagamenti senza disporre di un nodo
personale. L'utente deve possedere una copia degli header della catena
più lunga (che può ottenere dai nodi della rete) e ottenere il ramo di
Merkle che collega la transazione al blocco in cui è stata inserita. Non
può verificare la transazione da solo calcolando gli hash (non ha le
altre foglie dell'albero di Merkle), ma collegandola ad un punto
specifico della catena può vedere che un nodo l'ha accettata, ed
eventuali blocchi successivi confermano che anche l'intera rete l'ha
accettata.

\%TODO: grafico alto pag 5 satoshi

Così come è descritta, la verifica è affidabile fin tanto che la rete è
controllata da nodi onesti, ma è vulnerabile se la rete è soverchiata da
un attaccante. Mentre un nodo può effettuare le verifiche da se, il
metodo semplificato è vulnerabile alle transazioni ad-hoc create da un
attaccante che controlla la rete. Una strategia di difesa consiste
nell'accettare avvisi dai nodi della rete quando questi rilevano blocchi
non validi, richiedendo all'utente di scaricare l'intero blocco. Per
questo motivo chi utilizza bitcoin per business ritiene preferibile
avere un nodo personale con intera blockchain, in modo da poter
effettuare verifiche autonomamente e velocemente. \%FIXME: questo
paragrafo è scritto male

Da questo si può dedurre un fatto importante che distingue la rete
bitcoin dalle altre reti P2P: si può partecipare alla rete anche senza
il relativo software (in questo senso, si partecipa al livello
comunitario della rete): non serve essere un nodo, basta essere un
utente.

\section{Gestione dei valori}\label{gestione-dei-valori}

Anche se è possibile trattare le monete una ad una, è improponibile fare
una transazione per ogni singolo centesimo. Per la divisione degli
importi, le transazioni contengono molteplici input e output. In una
transazione normale si avranno o un singolo input proveniente da una
grande transazione precedente oppure più input provenienti da più
transazioni piccole precedenti, e al massimo due output: uno per il
pagamento vero e proprio e uno per restituire il resto, se presente, al
mittente.

Chiarimo meglio il concetto:

\begin{itemize}
\itemsep1pt\parskip0pt\parsep0pt
\item
  Un input è un riferimento ad un output di una precedente transazione
  che contiene l'indirizzo di chi sta effettuando la transazione. Se ci
  sono più input, l'importo degli output da loro referenziati viene
  sommato ed il totale è il massimo valore utilizzabile dall'output.
\item
  Un output contiene l'indirizzo del destinatario e il valore da
  spedire. Dato che, in una futura transazione, un output può essere
  referenziato da un solo input, potrebbe verificarsi il caso in cui
  l'input sia maggiore dell'output e della transaction fee desiderata.
  In questo caso è necessario creare due output, uno con il valore da
  spedire e l'indirizzo del destinatario, l'altro con la differenza da
  restituire al mittente, il \textbf{resto}. La differenza tra il totale
  degli input e il totale degli output è la transaction fee.
\end{itemize}

Nonostante la stretta dipendenza tra le varie transazioni, non è
necessario estrarre l'intero background di ogni input, in quanto la
transazione verrà accettata solo se il blocco che contiene le
transazioni con gli output referenziati dagli input è stato accettato
dalla rete.

Dato il valore elevato di una singola btc (che può variare da poche
decine a migliaia di dollari) e le difficoltà inerenti nel trattare
numeri in virgola mobile su un computer, l'unità di misura base della
transazione non è il btc ma il Satoshi \footnote{dal ``nome''
  dell'inventore di Bitcoin}, e 1 BTC = 100000000 Satoshi.

Questo sistema del calcolo del valore di una transazione è lo stesso
utilizzato dal sotftware che gestisce il portafoglio per calcolare il
proprio valore: per ogni indirizzo all'interno del portafogli, il
software scansiona le transazioni presenti nella blockchain che
contengono l'indirizzo in esame, somma i valori entranti nell'indirizzo,
sottrae quelli uscenti e ricava il valore ``contenuto'' nell'indirizzo.
Usando questo sistema, i possedimenti di un utente sono temporalmente
limitati all'ultimo blocco accettato nella catena, e non è quindi
possibile spendere moneta ricevuta da una transazione non ancora
approvata \footnote{tecnicamente, nessuna moneta è ricevuta fin tanto
  che la transazione non è approvata. L'utente non è nemmeno consapevole
  della transazione a lui destinata fino ad avvenuta approvazione.}.

\%\%\%\%\%\%\%\%\% \% La documentazione di Satoshi prevede anche privacy
e calcoli statistici sugli attacchi. \% Visto che prevedo una sezione
apposta per anonimato e sicurezza, ne parlo la e non qua. \%

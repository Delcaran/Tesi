\chapter{Scripting}\label{transaction-scripting}

Le transazioni Bitcoin contengono un potenziale nascosto che non viene ancora sfruttato nelle attuali implementazioni del client.
Alcuni campi delle transazioni possono infatti contenere piccoli script in un linguaggio di programmazione Forth-like basato su stack valutato da sinistra verso destra, e di fatto attualmente alcuni script standard vengono utilizzati per la verifica della correttezza delle transazioni. Il linguaggio di scripting è descritto in \cite{bitcoin-scripting-api} ed offre funzioni crittografiche quali \verb|SHA1|, che rimpiazza l'elemento in cima allo stack con il suo hash, e \verb|CHECKSIG| che estrae una chiave pubblica ECDSA e la relativa firma dallo stack, verifica la firma e carica il risultato (vero o falso) nello stack.\\\\
Per capire meglio, il seguente listato mostra il contenuto una transazione standard (transazione \verb|pay-to-pubkey-hash|) decodificato dall'esadecimale\footnote{Un esempio di messaggio di transazione non decodificato (così come una chiara rappresentazione di tutte le strutture dati usate dal protocollo) è disponibile in \url{https://en.bitcoin.it/wiki/Protocol_specification\#Transaction_Verification} } ed espresso in formato JSON\footnote{I dati sono qui stati troncati per brevità, ma la transazione è realmente esistente ed è visibile all'indirizzo \url{http://blockexplorer.com/rawtx/99383066a5140b35b93e8f84ef1d40fd720cc201d2aa51915b6c33616587b94f}}:

\lstinputlisting[tabsize=2, numbers=left]{src/transaction.json}

I campi della transazione hanno il seguente contenuto:

\begin{description}
    \item[\PVerb{hash}] è l'hash della transazione calcolato secondo appositi flag (vedere più avanti \ref{contratti}) e utilizzato per identificarla.
    \item[\PVerb{ver}] indica la versione del protocollo Bitcoin a cui questa transazione si riferisce.
    \item[\PVerb{vin_sz}] è il numero di input presenti.
    \item[\PVerb{vout_sz}] è il numero di output.
    \item[\PVerb{lock_time}] permette di rendere non pagabile la transazione (e quindi di modificarla) fino ad un certo istante nel futuro. Maggiori dettagli nella sezione \ref{contratti}.
    \item[\PVerb{size}] la dimensione in byte della transazione.
    \item[\PVerb{in}] array degli input.\begin{description}
      \item[\PVerb{prev_out}] contiene i riferimenti da cui prelevare le BTC:\begin{description}
            \item[\PVerb{hash}] è l'identificativo della transazione a cui fare riferimento.
            \item[\PVerb{n}] è l'indice dell'output della transazione di riferimento da cui prelevare le BTC.
        \end{description}
      \item[\PVerb{scriptSig}] contiene una porzione dello script usato per verificare la transazione. In questo caso è nella forma \verb|<sig> <pubKey>| dove \verb|sig| è la firma ECDSA dell'autore della transazione mentre \verb|pubKey| è la corrispondente chiave pubblica.
      \end{description}
    \item[\PVerb{out}] array degli output.\begin{description}
      \item[\PVerb{value}] ammontare di questo output.
      \item[\PVerb{scriptPubKey}] seconda porzione dello script. Si trova nella forma \verb|OP_DUP| \verb|OP_HASH160| \verb|<pubKeyHash>| \verb|OP_EQUALVERIFY| \verb|OP_CHECKSIG| dove \verb|pubKeyHash| è l'indirizzo del destinatario della transazione mentre le altre voci sono istruzioni del linguaggio di scripting.
    \end{description}
\end{description}

La verifica di un input di tale transazione avviene secondo la procedura illustrata nella tabella \ref{table:pay-to-pubkey-hash}, in cui la transazione da verificare verrà identificata come $\tau_1$, mentre la transazione precedente a cui si fa riferimento sarà $\tau_0$:

\begin{table}
  \centering
  \begin{tabular}{m{0.3\linewidth-2\tabcolsep} | m{\dimexpr 0.3\linewidth-2\tabcolsep} | m{\dimexpr 0.3\linewidth-2\tabcolsep}}
    \hline
    \textbf{Stack}&\textbf{Script}&\textbf{Descrizione} \\
    \hline
    \verb||&\verb|<sig>| \verb|<pubKey>| \verb|OP_DUP| \verb|OP_HASH160| \verb|<pubKeyHash>| \verb|OP_EQUALVERIFY| \verb|OP_CHECKSIG|&Unisco il campo \verb|scriptSig| di $\tau_1$ con il campo \verb|scriptPubKey| del relativo output di $tau_0$ creando lo script da eseguire  \\ \hline
    \verb|<sig>| \verb|<pubKey>|&\verb|OP_DUP| \verb|OP_HASH160| \verb|<pubKeyHash>| \verb|OP_EQUALVERIFY| \verb|OP_CHECKSIG|&Carico nello stack la firma dell'autore di $\tau_1$ e la relativa chiave pubblica \\ \hline
    \verb|<sig>| \verb|<pubKey>| \verb|<pubKey>|&\verb|OP_HASH160| \verb|<pubKeyHash>| \verb|OP_EQUALVERIFY| \verb|OP_CHECKSIG|&Duplico la chiave pubblica in cima allo stack. Si dovrà utilizzare due volte: prima per verificare che l'autore di $\tau_1$ sia il destinatario di $\tau_0$, e poi per verificare che possieda la chiave privata necessaria per spendere le BTC \\ \hline
    \verb|<sig>| \verb|<pubKey>| \verb|<pubHashA>|&\verb|<pubKeyHash>| \verb|OP_EQUALVERIFY| \verb|OP_CHECKSIG|&Calcolo l'hash della chiave pubblica dell'autore di $\tau_1$ in cima allo stack e lo carico nello stack stesso \\ \hline
    \verb|<sig>| \verb|<pubKey>| \verb|<pubHashA>| \verb|<pubKeyHash>|&\verb|OP_EQUALVERIFY| \verb|OP_CHECKSIG|&Carico nello stack la chiave pubblica del destinatario di $\tau_0$\\ \hline
    \verb|<sig>| \verb|<pubKey>|&\verb|OP_CHECKSIG|&Verifico che i due elementi in cima allo stack siano uguali, ovvero che l'autore di $\tau_1$ sia anche il destinatario di $\tau_0$ \\ \hline
    \verb|true|&\verb||&Verifico la firma (ovvero il possesso da parte dell'autore di $\tau_1$ della necessaria chiave privata) e salvo il risultato nello stack \\ \hline
  \end{tabular}
  \caption[Verifica di una transazione]{Passaggi della verifica dell'input di una transazione standard tra due utenti.\label{table:pay-to-pubkey-hash}}
\end{table}

L'altra transazione standard accettata dall'implementazione attuale è quella con cui si pagano le ricompense derivanti dal ritrovamento di un blocco, ovvero le transazioni \emph{coinbase} che appartengono alla tipologia \verb|pay-to-pubkey|. Non essendoci un indirizzo sorgente o una transazione in input, i valori dei campi sono più ridotti: \verb|scriptPubKey| contiene \verb|<pubKey> OP_CHECKSIG| mentre \verb|scriptSig| contiene unicamente \verb|<sig>|. Il funzionamento di questo script è descritto nella tabella \ref{table:pay-to-pubkey} e si usano le stesse convenzioni della tabella precedente.

\begin{table}
  \centering
  \begin{tabular}{m{0.3\linewidth-2\tabcolsep} | m{\dimexpr 0.3\linewidth-2\tabcolsep} | m{\dimexpr 0.3\linewidth-2\tabcolsep}}
    \hline
    \textbf{Stack}&\textbf{Script}&\textbf{Descrizione} \\
    \hline
    \verb||&\verb|<sig>| \verb|<pubKey>| \verb|OP_CHECKSIG|&Unisco \verb|scriptSig| da $tau_1$ e \verb|scriptPubKey| da $tau_0$ creando lo script da analizzare  \\ \hline
    \verb|<sig>| \verb|<pubKey>|&\verb|OP_CHECKSIG|&Carico nello stack la firma dell'autore di $tau_1$ e la chiave pubblica del destinatario di $tau_0$ \\ \hline
    \verb|true|&\verb||&Verifico la firma e salvo il risultato nello stack \\ \hline
  \end{tabular}
  \caption[Verifica di una transazione di generazione BTC]{Passaggi della verifica di una transazione standard di generazione di BTC.\label{table:pay-to-pubkey}}
\end{table}

Con il linguaggio di scripting è anche possibile effettuare alcune transazioni particolari non ancora considerate standard, per realizzare le quali è necessario utilizzare dei client appositi in grado di manipolare la struttura grezza delle transazioni. Ad esempio è possibile creare una transazione il cui output non è spendibile semplicemente inserendo all'inizio di \verb|scriptPubKey| il comando \verb|OP_RETURN|, il quale termina subito l'esecuzione dello script rendendo la transazione non utilizzabile in un futuro input in quanto il contenuto di \verb|scriptSig| diventa inutile. Esiste un esempio di tale tipo di transazione\footnote{\url{https://blockexplorer.com/tx/eb31ca1a4cbd97c2770983164d7560d2d03276ae1aee26f12d7c2c6424252f29}} che avendo l'unico output impostato a 0, lascia l'intero ammontare di 0.125 BTC come transaction fee al minatore che troverà il blocco.\\
Dal lato opposto, è anche possibile creare transazioni spendibili da chiunque\footnote{Un motivo per utilizzare in futuro questo tipo di transazioni è illustrato nel Wiki all'indirizzo \url{https://en.bitcoin.it/wiki/Fidelity_bonds}.} lasciando vuoto \verb|scriptPubKey| e inserendo \verb|OP_TRUE| in \verb|scriptSig|.\\
Un ulteriore esempio delle potenzialità dello scripting è la creazione di alcuni puzzle crittografici che devono essere risolti per poter spendere le BTC. Esiste un esempio \footnote{\url{https://blockexplorer.com/tx/a4bfa8ab6435ae5f25dae9d89e4eb67dfa94283ca751f393c1ddc5a837bbc31b}} in cui si richiede di trovare qualche dato tale che, sottoposto due volte ad hash, risulti uguale ad un dato fornito. La tabella \ref{table:puzzle-transaction} illustra l'esecuzione di un simile script.

\begin{table}
  \centering
  \begin{tabular}{m{0.3\linewidth-2\tabcolsep} | m{\dimexpr 0.3\linewidth-2\tabcolsep} | m{\dimexpr 0.3\linewidth-2\tabcolsep}}
    \hline
    \textbf{Stack}&\textbf{Script}&\textbf{Descrizione} \\
    \hline
    \verb||&\verb|<data>| \verb|OP_HASH256| \verb|<given_hash>| \verb|OP_EQUAL|&Unisco \verb|scriptSig| e \verb|scriptPubKey| creando lo script da analizzare  \\ \hline
    \verb|<data>|&\verb|OP_HASH256| \verb|<given_hash>| \verb|OP_EQUAL|&Carico nello stack \verb|scriptSig| \\ \hline
    \verb|<data_hash>|&\verb|<given_hash>| \verb|OP_EQUAL|&Calcolo l'hash del dato \\ \hline
    \verb|<data_hash>| \verb|<given_hash>|&\verb|OP_EQUAL|&Carico nello stack l'hash di confronto \\ \hline
    \verb|true|&\verb||&Confronto i due hash \\ \hline
  \end{tabular}
  \caption[Verifica di transazione con puzzle]{Passaggi della verifica di una transazione di esempio contenente un puzzle.\label{table:puzzle-transaction}}
\end{table}

\section{Contratti}\label{contratti}

I contratti sono un modo per sfruttare le potenzialità dello scripting Bitcoin per creare accordi tra persone in modo da limitare al minimo la fiducia da riporre in esse. Come per alcuni degli script precedentemente descritti, anche questi non sono considerati standard e sono anzi da considerare solo come esempi di una possibile futura evoluzione delle rete che permette nuove forme di transazioni evadibili in automatico tramite la blockchain senza necessità di intervento umano. Gli esempi sono presi dal Wiki ufficiale \cite{bitcoin-contracts} che a sua volta prende spunto da una dissertazione di Nick Szab\'{o} \cite{nick-szabo}.\\

I contratti vengono realizzati sfruttando alcune caratteristiche delle transazioni e del linguaggio di scripting che non sono considerate ancora standard.
Una di queste caratteristiche è il \emph{time-lock} visto in precedenza: ogni transazione può infatti possedere un blocco temporale che ne consente la modifica e l'eliminazione entro un certa tempo (ad esempio un timestamp o l'indice di un blocco) e se impedisce lo sfruttamento dei BTC coinvolti fino a quel momento. Una transazione con time-lock che ha superato l'istante previsto diventa definitiva.
Nel caso di modifica di una transazione viene incrementato un contatore interno (uno per ogni voce di input della transazione) fino ad un valore massimo. Per cui anche nel caso in cui il time-lock non sia stato raggiunto, se tutti i contatori sono arrivati al loro valore massimo (definito come \verb|UINT_MAX|), allora la transazione viene considerata definitiva. I contatori possono essere utilizzati per generare nuove versioni di una transazione senza invalidare le firme precedenti, ad esempio nel caso in cui si voglia aggiungere un input proveniente da un diverso utente mediante l'utilizzo dei flag precedentemente accennati. Tali flag \verb|SIGHASH| definiscono il modo in cui viene creato l'hash che viene utilizzato nella fase di verifica della firma e permettono dunque di costruire transazioni multi-utente in cui ciascuno degli utenti coinvolti firma una parte della transazione (il proprio input) senza dover (e poter) firmare o influenzare le altre parti.\\
I flag sono composti da due componenti, una modalità e un modificatore. Le modalità possibili sono tre:
\begin{description}
    \item[\PVerb{SIGHASH_ALL}:] è il valore di default, indica che tutta la transazione viene firmata ad eccezione degli script di input (che se venissero firmati anch'essi renderebbero impossibile la creazione della transazione, per cui sono sempre esclusi dalla firma). Vengono però incluse nella firma le altre proprietà dell'input quali gli output connessi e i contatori. È equivalente a firmare un contratto che reciti ``Sono d'accordo di impiegare danaro proveniente da questi miei fondi se tutti impiegano il danaro proveniente dai loro fondi e il totale impiegato è quello definito.''.
    \item[\PVerb{SIGHASH_NONE}:] gli output non vengono firmati e possono contenere qualsiasi cosa. Il contratto sarebbe ``Sono d'accordo di impiegare danaro proveniente da questi miei fondi fintanto che tutti impiegano il danaro proveniente dai loro fondi, ma non sono interessato a come verrà impiegata tale somma o se verrà usata nella sua totalità.''. Con questa modalità si autorizza altri utenti ad aggiornare la transazione modificando i contatori relativi ai propri input.
    \item[\PVerb{SIGHASH_SINGLE}:] come per la modalità precedente, gli input sono firmati ma non i contatori, in modo che altri utenti possano aggiornare la transazione. L'unico output che viene firmato è quello inserito nella transazione nella stessa posizione dell'input\footnote{Per chiarezza, tornando all'array degli input e degli output in formato JSON, l'output all'indice $x$ verrà firmato tramite l'input  all'indice $x$}. Il contratto equivalente è ``Accetto il contratto fintanto che il mio output sia quanto io ho stabilito, indipendentemente dagli altri firmatari.''.
\end{description}

Il modificatore è invece \verb|SIGHASH_ANYONECANPAY|, che, se inserito in uno script di input, indica che tale input viene firmato ma non è necessario che lo siano anche gli altri.\\
È inoltre possibile utilizzare più chiavi pubbliche tramite l'opcode \verb|CHECKMULTISIGVERIFY| tramite il quale si possono fornire più chiavi pubbliche e richiedere un numero minimo di firme che devono essere valide, numero che può anche essere minore del numero di chiavi fornite. Ad esempio un output può richiedere due firme per essere verificato con lo script \verb|2| \verb|<pubKey1>| \verb|<pubKey2>| \verb|2| \verb|CHECKMULTISIGVERIFY|.\\
Infine un aspetto fondamentale per realizzare un contratto è la tecnica di propagazione nella rete dello stesso: per essere valido (in termini legali, non in termini crittografici), il contratto deve essere visibile a tutti i possibili firmatari prima della firma, in modo che essi sappiano quale accordo stanno sottoscrivendo. Esistono due metodi generici per fare ciò in modo sicuro (ovvero con minima fiducia tra le parti):
\begin{itemize}
    \item Inviare le transazioni tra i possibili firmatari al di fuori della rete Bitcoin in forma incompleta o invalida e metterle in rete una volta complete.
    \item Utilizzare due transazioni. La prima (il contratto) viene creata e firmata ma non inoltrata nella rete. L'altra transazione (il pagamento) viene inoltrata dopo che il contratto è stato siglato in modo da bloccare le BTC, dopodiché viene inoltrato il contratto.
\end{itemize}

Tutti questi strumenti combinati possono portare alla creazione di nuovi strumenti finanziari che si basano sulla rete Bitcoin, come ad esempio quelli illustrati brevemente di seguito. Per maggiori dettagli o altri possibili contratti sono disponibili informazioni più approfondite in \cite{bitcoin-contracts,nick-szabo,bitter-better}.

\subsection{Deposito temporaneo}

Rimanendo nell'ambito della rete Internet, è possibile voler sottoscrivere un account presso un sito Internet che richieda una certa fiducia nell'utente, ad esempio per verificare che egli non sia uno spambot. Una soluzione possibile consiste nell'inviare una sorta di caparra all'operatore del sito, fidandosi del fatto che tale somma verrà restituita in caso di cancellazione dell'account e che l'operatore non scapperà con il denaro. Tale fiducia può essere eliminata con un contratto Bitcoin secondo la seguente procedura:
\begin{enumerate}
    \item Utente ed operatore si scambiano una coppia di chiavi pubbliche appositamente create.
    \item L'utente crea la transazione $\tau_1$ (il pagamento) contenente X BTC in un output che richiede la firma sia dell'utente che dell'operatore, ma non la inserisce in rete.
    \item L'hash della transazione appena creata viene inviata all'operatore.
    \item L'operatore crea una seconda transazione $\tau_2$ (il contratto) la quale paga l'output di $\tau_1$ all'utente. \verb|lock_time| viene impostato ad una qualche data nel futuro e il contatore viene inizializzato a 0. Dato che l'output di $\tau_1$ richiede la firma sia dell'operatore che dell'utente per essere speso, la transazione è incompleta.
    \item $\tau_2$ incompleta viene inviata all'utente, il quale verifica che il denaro prima o poi tornerà nelle sue tasche alla data designata, salvo modifiche. Essendo il contatore impostato a 0, il contratto può essere modificato in futuro se entrambe le parti sono concordi. Lo script in input di $\tau_2$ è incompleto visto che manca la firma dell'utente, che verrà apposta una volta verificato il contratto, completando così la transazione.
    \item L'utente può ora inviare $\tau_1$ per pagare l'operatore con la somma pattuita e successivamente inoltrare $\tau_2$ per rendere valido il contratto.
\end{enumerate}

Una volta completati tali passaggi, le X BTC inviate con $\tau_1$ si trovano in uno stato in cui non possono essere spese né dall'utente né dall'operatore senza l'approvazione dell'altro. Allo scadere del \verb|time_lock| il contratto si concluderà e l'utente riavrà indietro le sue monete anche nel caso in cui il sito o l'operatore siano spariti.\\

Nel caso in cui l'utente voglia chiudere anticipatamente il contratto, l'operatore crea una nuova versione di $\tau_2$, ad esempio $\tau_{2.1}$, in cui \verb|lock_time| è impostato a 0 e il contatore al suo massimo \verb|UINT_MAX| e applica la sua firma. $\tau_{2.1}$ viene inviata all'utente, il quale controlla, eventualmente firma e inoltra la transazione alla rete invalidando $\tau_2$ e riappropriandosi subito delle sue BTC.\\
La stessa procedura viene applicata se l'utente volesse prolungare il contratto: basta impostare un nuovo \verb|lock_time|, incrementare di una unità il contatore, firmare ed inoltrare.

\subsection{Acquisto di beni con mediatore}

La preoccupazione più grande di tutti coloro che acquistano online o con altri mezzi telematici è ``Il prodotto che ricevo (se lo ricevo) sarà quello per cui ho pagato? E se non sono soddisfatto riavrò i soldi indietro?''. Simile è anche la preoccupazione di chi vende nel caso in cui il pagamento venga effettuato dopo la spedizione o dopo aver dovuto effettuare alcune spese: ``Riceverò mai il denaro che mi spetta?''. Sfruttando piattaforme quali eBay o Amazon è possibile richiedere l'intervento di un mediatore (appunto, eBay o Amazon) per risolvere la disputa e assegnare il denaro a colui che ne ha diritto.\\
Sfruttando Bitcoin e le opzioni di firma multipla viste è possibile eseguire le stesse operazioni, ovvero creare un pagamento tra tre parti che rimane bloccato fintanto che due delle tre parti non decidono a chi spetta il denaro. La procedura è la seguente:
\begin{enumerate}
    \item Ognuna delle parti crea la sua chiave pubblica: $k_c$ per il commerciante, $k_a$ per l'acquirente e $k_m$ per il mediatore. Commerciante e mediatore inviano le proprie chiavi all'acquirente.
    \item L'acquirente invia al commerciante $k_m$.
    \item Il commerciante ``sfida'' il mediatore con una nonce casuale, che viene da lui firmata con la porzione privata di $k_m$, garantendo che tale chiave appartenga proprio al commerciante.
    \item L'acquirente crea la transazione $\tau_1$ nel cui output sarà presente lo script \verb|2| \verb|<Kc>| \verb|<Km>| \verb|<Ka>| \verb|3| \verb|CHECKMULTISIGVERIFY| e la inoltra nella rete.
\end{enumerate}

La situazione che si viene a creare blocca le monete inviate con la transazione in uno stato di non spendibilità fino a quando non avviene uno dei seguenti fatti:
\begin{itemize}
    \item Commerciante ed acquirente raggiungono un accordo (il commercio è andato a buon fine oppure il commerciante accetta di rimborsare il cliente).
    \item Acquirente e mediatore si accordano (il commercio è andato male e il mediatore dà ragione all'acquirente).
    \item Il mediatore spalleggia il commerciante (i beni sono comunque arrivati a destinazione come richiesto e il denaro spetta al commerciante).
\end{itemize}
Infatti l'output della transazione inoltrata, per essere speso, richiede che l'input della nuova transazione contenga almeno due delle firme collegate alle chiavi pubbliche elencate. Per cui una volta raggiunto l'accordo, una delle parti crea una $\tau_2$ con \verb|scriptSig| contenente la sua firma, la inoltra alla seconda parte interessata la quale la firma e la inoltra alla rete assegnando il denaro a chi di dovere.

\subsection{Raccolta fondi assicurata}

Si immagini un'associazione di volontari che intenda creare un bene comune, ad esempio gli abitanti di un quartiere che vogliono realizzare un parco giochi o la Protezione Civile che intende sistemare gli argini di un fiume senza intaccare gli scarsi fondi comunali, e che avvia appositamente una raccolta fondi per finanziare il progetto. Probabilmente tutti gli interessati investirebbero nell'impresa secondo le loro potenzialità, ma cosa succede se i fondi raccolti non raggiungono la somma preventivata per la buona riuscita dell'opera? Si cambia opera? Si abbassa la qualità del lavoro finale? Si restituiscono i soldi? E in quest'ultimo caso come si procede in caso di donazioni anonime?
Con una raccolta fondi assicurata ci si impegna a pagare una somma se e soltanto se il totale delle donazioni raggiunge una certa quota prefissata, altrimenti nessuno deve pagare nulla.
Si può implementare questa funzionalità come segue:
\begin{enumerate}
    \item Un imprenditore crea un nuovo indirizzo per la raccolta fondi e stima il traguardo a X BTC.
    \item Chiunque intenda contribuire crea una nuova transazione prendendo come input il valore che desidera inviare ma non la inoltra. La transazione differisce per tre particolari da una transazione normale: \begin{itemize}
        \item Non esiste resto. L'utente deve quindi effettuare una transazione preventiva per poter creare un output dell'ammontare desiderato da usare come input per la donazione nel caso non ne possieda già uno.
        \item Lo script di input è firmato con ``\verb|SIGHASH_ALL| $\|$ \verb|SIGHASH_ANYONECANPAY|``.
        \item Il valore in output è esattamente X BTC. Dato che tale output è maggiore della somma degli input, la transazione non risulta valida ed è quindi totalmente sicuro inviarla ad uno sconosciuto.
    \end{itemize}
    \item La transazione viene caricata sul server dedicato alla raccolta fondi. L'imprenditore tiene conto quanto essa vale e aggiorna il valore ancora mancante per il traguardo.
    \item Appena raggiunto il traguardo, tutte le transazioni invalide inviate vengono unite in una unica transazione. Tale transazione avrà come input tutti gli input delle donazioni e come output l'indirizzo della raccolta fondi, ovvero lo stesso presente in modo identico in tutti gli output delle donazioni.
    \item La transazione così creata viene inoltrata nella rete, effettuando quindi i pagamenti.
\end{enumerate}
L'utilizzo di \verb|SIGHASH_ALL| è il comportamento predefinito in una transazione standard Bitcoin e indica che viene firmato tutto il contenuto della transazione tranne lo script di input. Nel caso specifico significa che non è possibile per il realizzatore della raccolta fondi modificare la transazione di riferimento dell'input e quindi nemmeno l'ammontare che l'autore della transazione desidera donare. Il modificatore \verb|SIGHASH_ANYONECANPAY| indica che la firma indicata nell'input della transazione si riferisce solo ed esclusivamente a quell'input e non deve essere presa in considerazione quando si tratteranno altri input.
Combinando questi due opcode si ottiene una transazione che risulta ancora perfettamente valida nel caso in cui vengano aggiunti altri input ma risulta invalida se viene modificata una qualsiasi altra parte di essa. Si ha quindi la garanzia che, se la somma che si intende donare viene spesa, finisce esattamente nell'indirizzo desiderato e nell'ammontare desiderato e ciò è avvenuto perché si è raggiunta la somma necessaria all'impresa. E' da notare come se le donazioni sono in eccesso rispetto alla somma necessaria, la differenza diventa una transaction fee destinata allo scopritore del blocco in quanto non è possibile modificare gli output in modo da dirigere tale differenza verso un secondo indirizzo magari dedito alla stessa causa. L'imprenditore che ha avviato la raccolta fondi deve quindi prestare attenzione alle somme donate e combinarle in modo da produrre l'output desiderato, tasse incluse.

\section{Sviluppi}

Le funzionalità di scripting sono ancora un campo tutto sommato inesplorato del protocollo. Molte funzionalità sono state rimosse, aggiunte o modificate nel corso nel tempo trasformando tipologie di transazioni inizialmente standard in transazioni non più propagate dai nodi. Tale mancanza di stabilità ha notevolmente influito sulla diffusione della moneta, invalidandone alcune potenzialità decisamente interessanti. Gli esempi sopra riportati sono solamente alcuni dei molti teorizzati o implementati in versioni personalizzate dei client: tali versioni non standard non vengono però riconosciute dai nodi che utilizzano il client ufficiale, e sono quindi sconsigliate al di fuori di un ambito particolarmente ristretto e fortemente controllato.


\newglossaryentry{btc}{
	name=BTC,
	description={Bitcoin intesa come unità di misura della moneta elettronica}
}

\newglossaryentry{satoshi}{
	name=satoshi,
	description={Unità di misura del valore di una transazione utilizzata all'interno della struttura dati. 1 \gls{btc} = 100000000 satoshi}
}
\newglossaryentry{churm}{
	name=churm,
	description={Indica l'inserimento o la rimozione di un nodo o di un file in una rete P2P}
}

\newglossaryentry{pdf}{
	name={Funzione di Densità di Probabilità},
	description={Funzione che descrive la probabilità che una variabile casuale assuma un certo valore. Per un range di valori, è il risultato dato dall'integrazione della funzione rappresentante il valore in quell'intervallo.},
	first={Funzione di Densità di Probabilità (PDF)},
	text={funzione di densità di probabilità}
}

\newglossaryentry{cdf}{
	name={Funzione di Ripartizione},
	description={Descrive la probabilità che una variabile casuale con una data \gls{pdf} assuma valori minori o maggiori ad $x$. Si calcola integrando la funzione che rappresenta la variabile nell'intervallo $[-\infty, x]$ oppure $[x, \infty]$.},
	first={funzione di ripartizione (CDF)},
	text={funzione di ripartizione}
}

\newglossaryentry{processopoisson}{
  name={Processo di Poisson},
  description={Un qualsiasi esperimento in cui si simula e si conta il quanti specifici eventi stocastici indipendenti avvengono durante la durata della rilevazione. Il numero di eventi che si sono manifestati entro due intervalli di tempo segue la \gls{distribuzionepoisson}},
  first={processo di Poisson},
  text={processo di Poisson}
}

\newglossaryentry{distribuzionepoisson}{
  name={Distribuzione di Poisson},
  description={Esprime le probabilità per un numero di eventi indipendenti che si verificano successivamente in uno specifico intervallo temporale sapendo quanti se ne verificano mediamente.\\Definendo $\lambda$ come il numero medio di eventi in analisi si ha 
  $$P(n) = e^{-\lambda}\frac{\lambda^n}{n!}$$},
  first={distribuzione di Poisson},
  text={distribuzione di Poisson}
}

\newacronym{nmi}{NMI}{Normalized Mutual Information}
\newacronym{ami}{AMI}{Adjusted Mutual Information}
\newacronym{hac}{HAC}{Hierarchical Agglomerative Clustering}
\newacronym{kmc}{KMC}{K-Means Clustering}

\newacronym{dag}{DAG}{Grafo Direzionale Aciclico}

\newglossaryentry{gof}{
	name={Goodness-of-Fit},
	description={Descrive quanto bene un modello statistico si adatta ai dati osservati. La metodologia di Kolmogorov-Smirnov consiste nel comparare un campione dei dati con una distribuzione di probabilità}
}

\newglossaryentry{pvalori}{
	name={p-valori},
	description={Dato un test statistico, ovvero un riassunto numerico che riduce un insieme di dati ad un unico valore, è la probabilità di ottenere un secondo test statistico almeno estremo quanto il primo}
}

\newglossaryentry{weaklyconnected}{
	name={Grafo Debolmente Connesso},
	description={Un grafo direzionato è debolmente connesso quando sostituendo i suoi archi con archi non diretti si produce un grafo connesso}
}

\newglossaryentry{component}{
	name={Componente Connessa},
	description={Sottografi massimali che contengono tutte le componenti connesse tra loro}
}

\newglossaryentry{biconnected}{
	name={Grafo biconnesso},
	description={Un grafo si dice biconnesso quando la rimozione di un nodo mantiene il grafo connesso}
}

\newacronym{amlcft}{AML-CFT}{Anti-Money-Laundering and Combating the Financing of Terrorism}

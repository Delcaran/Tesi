\newglossaryentry{btc}{
	name=BTC,
	description={Bitcoin intesa come unità di misura della moneta elettronica}
}

\newglossaryentry{componenteconnessa}{
  name={Componente Connessa},
  description={Dato G=(V,E) un grafo non orientato, una componente connessa è un sottografo massimale di un grafo G tale che per ogni coppia di nodi distinti del sottografo esiste un cammino che porta dal primo al secondo nodo (e viceversa).}
}

\newglossaryentry{componentebiconnessa}{
  name={Componente Biconnessa},
  description={Dato G=(V,E) un grafo non orientato, una componente biconnessa è un sottografo massimale di un grafo G tale che il sottografo è connesso e se viene rimosso un nodo e gli archi a lui incidenti, il sottografo risultante è ancora connesso.},
  plural={componenti biconnesse}
}

\newglossaryentry{debolmenteconnessa}{
  name={Componente Debolmente Connessa},
   description={Un grafo orientato si dice debolmente connesso se due qualsiasi vertici sono uniti da un cammino non orientato.},
   text={componente debolmente connessa},
   plural={componeneti debolmente connessi}
}

\newglossaryentry{fortementeconnessa}{
  name={Componente Fortemente Connessa},
   description={Una componente fortemente connessa di un grafo diretto G è un sotto-grafo massimale di G in cui esiste un cammino orientato tra ogni coppia di nodi ad esso appartenenti. Un nodo non può trovarsi contemporaneamente in due componenti fortemente connesse. Un grafo diretto è fortemente connesso se e solo se ha una sola componente connessa.},
   text={componente fortemente connessa},
   plural={componenti fortemente connessi}
}

\newglossaryentry{satoshi}{
	name=satoshi,
	description={Unità di misura del valore di una transazione utilizzata all'interno della struttura dati. 1 \gls{btc} = 100000000 satoshi}
}
\newglossaryentry{churn}{
r	name=churn,
	description={Indica l'inserimento o la rimozione di un nodo o di un file in una rete P2P}
}

\newglossaryentry{pdf}{
	name={Funzione di Densità di Probabilità},
	description={Funzione che descrive la probabilità che una variabile casuale assuma un certo valore. Per un range di valori, è il risultato dato dall'integrazione della funzione rappresentante il valore in quell'intervallo.},
	first={Funzione di Densità di Probabilità (PDF)},
	text={funzione di densità di probabilità}
}

\newglossaryentry{randomwalk}{
  name={Passeggiata Aleatoria Binomiale},
  description={Rappresenta l'idea di procedere in direzioni casuali. Formalmente, una variabile aleatoria discreta $X(N)$ fornisce , ad esempio, il numero di \emph{testa} usciti dopo $N$ lanci. Nel caso di 2 possibili scelte, come in questo caso, $X(N)$ ha una distribuzione binomiale, da cui il nome.},
  text={passeggiata aleatoria binomiale}
}

\newglossaryentry{provabernoulli}{
	name={Processo di Bernoulli},
	description={Si tratta di un processo aleatorio discreto in cui le variabili sono indipendenti ed aventi tutte la medesima legge di probabilità. Il tipico processo di Bernoulli consiste nel lanciare un numero infinito di volte una moneta non truccata, dove ogni lancio è una prova di Bernoulli e non fornisce nessuna indicazione sul risultato dei futuri lanci.},
	text={prova di Bernoulli}
}

\newglossaryentry{gradonodo}{
  name={Grado di un nodo},
  description={Numero di connessioni entranti o uscenti da un nodo in un grafo.},
  text={grado}
}

\newglossaryentry{leggepotenza}{
  name={Legge di Potenza},
  description={Si tratta di una qualsiasi relazione del tipo $f(x) = ax^k + o\left(x^k\right)$ dove $k$ è costante e $o\left(x^k\right)$ è una funzione asintoticamente piccola di $x^k$.},
  text={legge di potenza}
}

\newglossaryentry{cdf}{
	name={Funzione di Ripartizione},
	description={Descrive la probabilità che una variabile casuale con una data \gls{pdf} assuma valori minori o maggiori ad $x$. Si calcola integrando la funzione che rappresenta la variabile nell'intervallo $[-\infty, x]$ oppure $[x, \infty]$.},
	first={funzione di ripartizione (CDF)},
	text={funzione di ripartizione}
}

\newglossaryentry{processopoisson}{
  name={Processo di Poisson},
  description={Un qualsiasi esperimento in cui si simula e si conta il quanti specifici eventi stocastici indipendenti avvengono durante la durata della rilevazione. Il numero di eventi che si sono manifestati entro due intervalli di tempo segue la \gls{distribuzionepoisson}},
  first={processo di Poisson},
  text={processo di Poisson}
}

\newglossaryentry{distribuzionepoisson}{
  name={Distribuzione di Poisson},
  description={Esprime le probabilità per un numero di eventi indipendenti che si verificano successivamente in uno specifico intervallo temporale sapendo quanti se ne verificano mediamente.\\Definendo $\lambda$ come il numero medio di eventi in analisi si ha $P(n) = e^{-\lambda}\frac{\lambda^n}{n!}$},
  first={distribuzione di Poisson},
  text={distribuzione di Poisson}
}

\newacronym{nmi}{NMI}{Normalized Mutual Information}
\newacronym{ami}{AMI}{Adjusted Mutual Information}
\newacronym{hac}{HAC}{Hierarchical Agglomerative Clustering}
\newacronym{kmc}{KMC}{K-Means Clustering}

\newacronym{dag}{DAG}{Grafo Direzionale Aciclico}

\newglossaryentry{gof}{
	name={Goodness-of-Fit},
	description={Descrive quanto bene un modello statistico si adatta ai dati osservati. La metodologia di Kolmogorov-Smirnov consiste nel comparare un campione dei dati con una distribuzione di probabilità}
}

\newglossaryentry{pvalori}{
	name={p-valori},
	description={Dato un test statistico, ovvero un riassunto numerico che riduce un insieme di dati ad un unico valore, è la probabilità di ottenere un secondo test statistico almeno estremo quanto il primo}
}

\newglossaryentry{weaklyconnected}{
	name={Grafo Debolmente Connesso},
	description={Un grafo direzionato è debolmente connesso quando sostituendo i suoi archi con archi non diretti si produce un grafo connesso}
}

\newglossaryentry{component}{
	name={Componente Connessa},
	description={Sottografi massimali che contengono tutte le componenti connesse tra loro}
}

\newglossaryentry{biconnected}{
	name={Grafo biconnesso},
	description={Un grafo si dice biconnesso quando la rimozione di un nodo mantiene il grafo connesso}
}

\newacronym{amlcft}{AML-CFT}{Anti-Money-Laundering and Combating the Financing of Terrorism}

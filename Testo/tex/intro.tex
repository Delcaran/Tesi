\chapter{Introduzione}

P2P come alternativa a client-server "puro"
Esempio di una ricerca MP3 \cite{kurose_ross}
Chiarire come ogni peer è contemporaneamente sia un client che un server, in tal senso il paradigma è sempre client-server: esempio: se scarico via http, ogni peer è sia un client web che un web server temporaneo (temporaneo perché IP non è fisso).

\section{Distribuzione File}
Ciascun peer distribuisce la quantità di contenuti in suo possesso mentre scarica da altri peer quello che gli manca
\subsection{Scalabilità}
\begin{description}
\item[$u_s$]frequenza di upload verso il server
\item[$u_i$]frequenza di upload dell'$i-$esimo peer
\item[$d_i$]frequenza di download dell'$i-$esimo peer
\item[$F$]dimensione in bit del file da distribuire
\item[$N$]numero di peer che vuole una copia del file
\item[$D_{cs}$]tempo di distribuzione del file per l'architettura client-server
\end{description}
Implicazione: condizioni ottimali di distribuzione (rete dedicata)
\subsubsection{Client-Server}
Osservazioni:
\begin{itemize}
\item Il server deve trasmettere il file a $N$ peer, quindi $NF$ bit. Data la frequenza di upload $u_s$, il tempo per distribuire il file deve essere almeno $NF/u_s$.
\item sia $ d_{min} = \min\{d_1,d_p,\cdots,d_N \}$ la frequenza di download del peer con il valore più basso. Tale peer riceverà il file in almeno $F/d_{min}$ secondi, che è quindi il tempo minimo di distribuzione.
\end{itemize}
Da cui
\begin{displaymath}
D_{cs} \geq \max \left\lbrace \frac{NF}{u_s}, \frac{F}{d_{min}} \right\rbrace
\end{displaymath}
Questo è il limite inferiore al tempo di distribuzione minimo per l'architettura client-server. Trattiamo il caso ottimo e consideriamolo come il tempo di distribuzione effettivo, ovvero:
\begin{equation}
D_{cs} = \max \left\lbrace \frac{NF}{u_s}, \frac{F}{d_{min}} \right\rbrace
\end{equation}

\section{Localizzazione dei contenuti}
\subsection{Directory centralizzata}
Si usa un servizio centrale per fornire un servizio di directory, che i peer contattano per sapere quali peer hanno quali file e per rendere disponibili agli altri peer i propri file.
Il server usa una connessione TCP permanente con ogni peer o invia pacchetti per sapere quando questo va offline
\subsubsection{Svantaggi}
\begin{description}
    \item[Unico punto di rottura]
    \item[Collo di bottiglia delle prestazioni]
    \item[Violazione del copyright]
\end{description}


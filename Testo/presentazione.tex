\documentclass[italian]{beamer}
%\documentclass[italian,handout]{beamer}

\usepackage[utf8]{inputenc}

\title[Bitcoin]{Analisi di Bitcoin}
\subtitle{Anonimato, Sicurezza e Sviluppi Futuri}
\author[Paoluzzi Matteo]{Paoluzzi Matteo}
\institute[UniUD]{Università degli Studi di Udine\and{}Relatore:\\{}Prof. Ivan Scagnetto}
\date[2014/04/03]{IV Sessione di Laurea AA 2012/2013, Aprile 2014}
\subject{Bitcoin}

\usepackage{default}
\usepackage{pres_commands}
%\usepackage{pres_uniudtesi}
\usetheme{Madrid}

\graphicspath{{./img/}}

\hypersetup{
  pdfstartview={Fit},
  pdftitle={Presentazione della Tesi ``Analisi di Bitcoin: Anonimato, Sicurezza e Sviluppi Futuri``},
  pdfauthor={Paoluzzi Matteo},
  pdfsubject={Modello di presentazione di Tesi al computer},
  pdfkeywords={LaTeX pdf presentazione tesi laurea}}

\begin{document}

\frame{\titlepage}

\begin{frame}{Di cosa si tratta}
\begin{itemize}
 \item È una valuta utilizzabile per scambi commerciali.
 \item Esiste solo in forma elettronica.
 \item Non ha confini geografici.
 \item Bitcoin indica la rete e il protocollo.
 \item L'unità monetaria si indica con BTC.
\end{itemize}
\end{frame}

\begin{frame}{Criteri di progettazione}
\begin{itemize}
 \item Struttura completamente decentralizzata formata da nodi in un grafo casuale.
 \item Anonimato dell'utente.
 \item Indipendente da istituti centrali.
 \item Transazioni sicure e verificabili.
\end{itemize}
\end{frame}

\begin{frame}{Implementazione}
\begin{itemize}
 \item Si usa la cifratura a chiave pubblica per creare una serie di indirizzi in cui verrà depositato il denaro.
 \item Ogni transazione viene firmata con la chiave privata del mittente e contiene la chiave pubblica del destinatario.
 \begin{itemize}
    \item Solo il destinatario di una transazione può spendere il denaro trasferito come unico proprietario della chiave privata richiesta.
    \item Si viene a creare una catena di transazioni legate dalle chiavi pubbliche e private di mittente e destinatario.
 \end{itemize}
 \item Tutte le transazioni vengono rese pubbliche e confrontate tra loro in modo che nessuno possa spendere lo stesso denaro due volte.
 \item Le transazioni vengono fissate in blocchi che vengono concatenati tra loro.
\end{itemize}
\end{frame}

\begin{frame}{Sicurezza: le transazioni}
\begin{itemize}
    \item L'input di una transazione può solo essere l'output di una precedente transazione.
    \item Non è possibile spendere denaro che non è stato inviato ad un indirizzo di cui non si possiede la chiave privata.
    \item Le transazioni sono bloccate nel tempo da un timestamp e da un hash che le identifica in modo permanente.
    \item La rete verifica che uno stesso input non sia stato inviato a diversi output: un tentativo di attacco doppia-spesa.
\end{itemize}
\end{frame}

\begin{frame}{Sicurezza: i blocchi}
I blocchi vengono creati risolvendo un difficile problema crittografico.
\begin{itemize}
  \item Trovare l'hash di alcuni dati in modo che risulti un valore esadecimale inferiore ad uno specifico target.
  \item Il target viene ricalcolato ogni 2016 blocchi mantenendo fisso a 10 minuti il tempo medio per trovare un blocco.
  \item Tra i dati di cui calcolare l'hash ci sono le transazioni e l'hash del blocco precedente.
  \begin{itemize}
    \item La modifica di un blocco richiede la modifica di tutti i blocchi successivi.
    \item Concatenare blocchi rende sempre più difficile modificare un blocco vecchio.
    \item La catena di blocchi usata per fissare tutte le transazioni prende il nome di Blockchain.
  \end{itemize}
\end{itemize}
\end{frame}

\framedgraphic{Infografica}{transactions_ABC}

\begin{frame}{Vulnerabilità: velocità di transazioni}
 \begin{itemize}
  \item Un transazione non è confermata fintanto che non viene inserita in un blocco, il che richiede in media 10 minuti.
  \item Una transazione non confermata potrebbe non essere inserita nella blockchain e quindi risultare scartata.
  \item Un attacco doppia-spesa sfrutta questa debolezza per frodare un venditore poco accorto: \pause
  \begin{enumerate}
   \item Si crea una transazione legittima e la si invia al commerciante. \pause
   \item Il commerciante si fida che la transazione vada bene e invia il prodotto all'attaccante. \pause
   \item L'attaccante invia una seconda transazione che sovrascrive la prima inviando lo stesso denaro a se stesso, frodando il commerciante. \pause
   \item Se la seconda transazione viene inserita in un blocco prima della transazione legittima, l'attacco ha successo e l'attaccante ha ottenuto un bene senza pagarlo.
  \end{enumerate}
 \end{itemize}
 \pause
 La prevenzione è sempre nelle mani dell'utente.
\end{frame}

\begin{frame}{Vulnerabilità: potenza di calcolo}
 Se un utente riesce ad accumulare più potenza di calcolo di quella del resto della rete, potrebbe creare una nuova blockchain a sua discrezioni invalidando tutte le transazioni a partire da un momento a sua discrezione.\\
 \pause
 La difficoltà nel compiere tale operazione è elevata.
 \pause
 Ma questo non vuol dire che sia impossibile!
\end{frame}


\begin{frame}{Mining}
Creare un blocco ha molteplici scopi.
\pause
\begin{itemize}
  \item Blocca le transazioni in modo permanente.
  \item Permette la creazione di nuova moneta.
  \begin{itemize}
    \item Il lavoro speso per trovare il blocco viene ricompensato con BTC create ``dal nulla''.
    \item Il numero di BTC si dimezza nel tempo per mantenere il numero massimo di BTC a circa 21 milioni.
    \item Inizialmente il premio era di 50 BTC, ora è sceso a 25 BTC.
  \end{itemize}
\end{itemize}
\bigskip
\pause L'analogia è con i cercatori d'oro, che guadagnano dal ritrovamento delle pepite dopo ore di duro lavoro in miniera. \\
\pause Pertanto il processo di calcolo dell'hash di un nuovo blocco viene definito \emph{mining} e il nodo \emph{miner}.
\end{frame}

\begin{frame}{Mining Pools}
  Calcolare l'hash per un blocco è un evento casuale senza memoria che richiede elevata potenza di calcolo per essere portato a termine in tempi rapidi.
 \begin{itemize}
  \item Un computer end-user ha decisamente poche speranze di trovare un blocco e intascare la ricompensa.
  \item Con del costoso hardware dedicato le speranze aumentano un poco, ma restano comunque basse.
 \end{itemize}
 \bigskip
 \pause La soluzione è una collaborazione tra gli utenti in quelle che vengono definite \emph{mining pools}.
\end{frame}

\begin{frame}{Mining Pools}
  Mantenendo l'analogia con i minatori, le mining pools sono compagnie minerarie.
 \begin{itemize}
  \item Più minatori si cimentano in contemporanea nel ritrovamento di un blocco.
  \item Ad ogni blocco trovato la ricompensa viene divisa tra tutti coloro che hanno collaborato.
  \item Ogni pool ha un suo sistema di retribuzione con i suoi vantaggi e svantaggi.
 \end{itemize}
\end{frame}

\begin{frame}{Scripting}
Il protocollo sfrutta un linguaggio di scripting che permette di verificare le transazioni ma anche di crearne alcune che implementano situazioni diverse dal semplice pagamento.
\begin{itemize}
\item Sistemi di deposito temporaneo.
\item Raccolte fondi con assicurazione.
\item Acquisto di beni con un mediatore.
\end{itemize}
\bigskip
\pause
Essendo non-standard, attualmente queste transazioni vengono rifiutate dai nodi e non possono entrare a far parte della blockchain.\\
Possono però essere usare in reti private con client ad-hoc.
\end{frame}l

\begin{frame}{Deflazione}
 L'elevato valore di una singola BTC rende più attraente l'idea di accumularle invece che di spenderle. Questo è vero per almeno il 55\% delle BTC prodotte. L'accumulazione di BTC provoca la seguente catena di eventi:
 \begin{enumerate}
  \item Meno transazioni. \puase
  \item Meno blocchi nell'unità di tempo. \puase
  \item Meno nuove monete. \puase
  \imte Meno motivazione a diventare miner. \puase
  \item Meno utenti che verificano la correttezza delle transazioni. \puase
  \item Maggiore debolezza ai devastanti attacchi basati su potenza di calcolo e verifiche di transazioni. \puase
 \end{enumerate}
 Bitcoin è fortemente dipendente da una comunità attiva e da un costante fluire della moneta: in mancanza di ciò, perde il suo valore avviando potenzialmente una spirale deflazionistica.
\end{frame}



%%%%%%%%%%%%%%%%%%%%%%%%%%%%%%%%%%%%%%%%%%%%%%%%%%%%%%%%%%%%%%%%%%%%%%%%%%%%%%%%%%%%
% \item Formula che appare un poco per volta:
% \pause
% \parstepwise{
% $$
%   1\step{{}+2}
%   \step{{}+3} \step{{}+4}\step{{}+\cdots+n}
%   \step{{}={}}
%   \step{\sfondogiallo{$\displaystyle\frac{n(n+1)}{2}$}}
% $$
% }
%
%\pageTransitionGlitter{0}
% il comando \newframe e' come \newpage,
% ma non avanza il numero di pagina.
% Puo' servire per fare cambiamenti incrementali
% a una pagina, quando \pause o \stepwise non
% bastano. In questo esempio \pause non va bene
% perche' qui bisogna aggiungere un paragrafo
% ma allo stesso tempo  cambiare la figura che
% sta in cima alla pagina. Macchinoso da scrivere,
% ma puo' valerne la pena.
%
%\newframe
%
%\sfondogiallo{\rosso{\textit{coincidono:}}}
%
% Le transizioni si attivano al /pause
%\pageTransitionWipe{0}
%\pageTransitionWipe{180}
%\pageTransitionSplitVO
%\item {\setlength{\baselineskip}{2\baselineskip}
%Vedere la documentazione del pacchetto \textcolor{darkorange}{\texttt{texpower}}.
%}
%\pageTransitionReplace
%\pageTransitionBoxI
%%%%%%%%%%%%%%%%%%%%%%%%%%%%%%%%%%%%%%%%%%%%%%%%%%%%%%%%%%%%%%%%%%%%%%%%%%%%%%%%%%%%
\end{document}